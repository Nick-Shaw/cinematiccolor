\section{The goal}

Professionals in the motion picture, visual effects, animation, and games industries encounter color management challenges which are not covered in either traditional color science textbooks or online resources, leaving digital artists and computer graphics developers to fend for themselves. Best practices are too often maintained through tribal knowledge: passed along by word of mouth, user forums, or scripts copied between facilities. This document’s goal is to provide a better mechanism for communicating this information.

The paper also outlines the color pipeline challenges in modern feature film, visual effects, animation and games production, from on-set capture to the computer graphics pipeline to digital intermediate (DI) grading and final display. It presents techniques and best practices currently in use at major production facilities and notes areas that need further development and standardization.

One note to begin with: Color pipelines are not static nor is there a single best approach. They grow and evolve with the constraints of each project. This paper intends to present color measurement, processing, and pipelines in enough generality that the reader will be able to address the specific needs of their productions, even if they are not described explicitly in the text. It is unlikely that this paper will address the exact requirements of any given production in whole.

