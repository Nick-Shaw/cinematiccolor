\section{DI Grading}

Locking down the eventual DI viewing transform(s) early on is the critical first step in crafting cinematic color pipelines, as it pins down every other part of the process. The remainder of the pipeline essentially relies on visualizations that are created by working backwards from this step. 
Renders by Sony Pictures Imageworks, available for download at opencolorio.org. Images from “Cloudy With a Chance Of Meatballs” Courtesy of Sony Pictures Animation. © 2009 Sony Pictures Animation Inc. All rights reserved.
Digital intermediate (DI) grading is the process where the entire motion-picture is loaded into a digital device, for the purpose of color-grading in an environment that exactly mirrors the final exhibition.  Although the colorist may get involved much earlier in the project, DI grading is usually the final stage of post-production, where per-shot color grading is added and the visual look of the film is finalized and baked in. Color timing is the term used for traditional film lab color adjustments. The term Color correction is generally avoided as it implies that the original cinematography or VFX was flawed. However, correction is still used by colorists to mean continuity adjustments. The final step of baking in the Look, view transforms specific to different output devices plus any trims applied as a result of those transforms, is known as Mastering. As with so many terms encountered in this document the language is used imprecisely and often one term is substituted for another without the understanding of the finer implications. Whilst DI is generally understood to mean the color grade and mastering stage, it technically includes the entire post-production process and dates back to the 1990s when Kodak, who coined the phrase, were creating digital intermediate scenes for VFX then recording them back to film and cutting them into the intermediate negative, aka the interneg. Hence the term digital intermediate.

The DI grade often follows a conform of the edit decisions, which are usually made from proxies, using original source material and scene referred final effects. The conform includes final sound and titles, though these may not be available at the beginning of the session. When time is more important than quality or when the edit decision list cannot be exported, the DI grade can be applied to an export from the edit system. This is a major compromise and not recommended.

In games, the tasks and roles of the development team are not so delineated. There is rarely a specialist colorist involved and hence no DI grading session. However, games use the same or similar tools as DI grading integrated into the creation process. The nature of most games engines mandates a scene-referred workflow with a choice of output transforms to support play on any device. Consequently, Look Management is often part of the engine and far less of an issue than in the other disciplines. However there is still the concept of a mastering process, involving final quality adjustments in a review suite on one or more reference displays.

There are two main approaches to handling color in the DI grade. 
Unmanaged Output Referred 
Color Managed Scene Referred

The reference display must always be accurately calibrated and checked regularly. Color grade decisions should be made in real time to avoid chromatic adaptation induced errors. 

The first approach, unmanaged, is output-referred and began in the days of telecine and hardware grading systems. A telecine is a device that transfers film to video in real time. Telecines predate video tape recorders when the only way of recording or playing back recorded images was to use film. In the early days, recorded programs were broadcast from live telecine transmissions, but as the equipment and the standards improved, programming interfaces were added so that scene by scene grading could be applied. Programmed grades were mostly recorded to video tape recorders (VTRs) so the output was always a real-time video stream. There was no need for color management since there was only ever a single playout format and the grading accuracy was wholly dependent on the monitor calibration. The calibrated telecine monitor was often referred to as the “god monitor” so its importance was always very clear. In the modern version of unmanaged grading, imagery is loaded into a non-linear timeline with real-time playback capability; no viewing transform is applied. The image is manipulated in the grading system, akin to a motion-picture version of Photoshop, to look correct on the calibrated reference display. The grading system and the colorist using it, treat all media, video or log, in the same way, and decisions are all output-referred. Different camera image processing, exposure, transfer functions and VFX elements are generally matched and graded by eye rather than analytically. For those used to a Color Managed workflow, it might seem strange that anyone would work this way, but for decades this was the only possible approach and the resistance to change in our industry cannot be underestimated. The output-referred method is strongly discouraged. However, it it is sometimes used if there is insufficient data about the color gamut and the transfer function of the material on the timeline. Using incorrect color science can cause distortion, artifacts and clipping. See 3.3.1 Camera profiles for more detail. Output-referred grading is also used in television and commercials where DI grading is not always the last process and there is typically a single delivery format. Scene-referred grading is considered slower, more demanding and generally more expensive in some markets and because they have less time, smaller budgets and shorter-term products they are slow to transition. When there is no color management in the final DI grading session, it is much more difficult to implement accurate Look Management. Nevertheless, preview LUTs from the colorist can be used on set and in editing or grades from the DIT or editor can be used as a reference for the DI colorist.

The second Color Managed approach is usually, but not always scene-referred. For the DI grade, the reference display is pivotal to the color managed transforms. The display format is defined by the delivery specification and when there are multiple deliverables the largest gamut for which there is a calibrated display is chosen. The timeline color space should be scene-referred and all sources mapped to it. 

Sometimes the output color space is used as the working color space and transforms are only applied to the sources. With this approach grading is output-referred, and there is no real benefit to it, so it is not discussed further in this paper.

\subsection{Conforming}

The dated video-centric approach to grading comes from the time when colorists worked on a reel of film or a tape. Film and tape sources playback as a fixed linear timeline and were usually graded out of context. Media was passed from color grading back to editorial for mastering. That workflow is still used especially for commercial work, although the tape source has now been replaced by a random access digital file. The file is broken down into shots or events either by loading an edit decision list (EDL) or by scene detecting the program. The workflow is not ideal for VFX inserts but in this scenario completed VFX files are added by the editor, prior to the color grade. Titles are added after the color grade. VFX are usually submitted as output-referred files that match the delivery format reference display. This allows very little manipulation in grading and can suffer a quality loss since the pipeline is not scientifically color managed. A more flexible alternative is to discuss the process with the editor and deliver VFX log encoded, preserving more dynamic range and having more latitude in the grade. Since the final look is unknown until after the DI grade, the VFX log image, just like camera media benefits from this extended dynamic range.
 
In the recommended workflow, the DI colorist receives the edit decision list, the original source material, VFX that match the source material and the final soundtracks, and then conforms them as a multi-track non-linear timeline in the grading system. The conform might be done by the colorist, an assistant or by an editor dedicated to the task. The colorist then matches the scenes, manages the look and applies shot to shot or frame to frame fixes as necessary. The best results are obtained from a scene-referred workflow that uses color managed transforms at each stage. This brings different sources closer together before the grading begins, but more importantly, it makes it much easier to deliver alternate formats using output transforms, which are more accurate and less work than using just trims. It also permits VFX, editorial and other departments to see representative images on displays calibrated to different standards, at any stage of the process.
 
The edit decision list could be as simple as one or more EDL files, one per track, but is usually an XML or AAF that supports multiple tracks. A reference movie of the cut should be provided to check for errors in conform. These are particularly likely to occur when size and speed changes are included in the edit decision list. This list is crucial in selecting shots for VFX and determining the exact frames required. VFX should be returned with frame handles of a consistent length which simplifies conforming if there are updates. The number of frames in the handles should be agreed and specified early in the production. Mattes generated at the time the VFX are rendered are best combined as an alpha channel or multi layer EXR. Mattes created later in post production for use in DI grading can be delivered as separate files. Most DI grading systems can parse separate mattes from the RGB and alpha channels of an image file. The mattes need to be synchronized with the effect shot so it is always recommended that the matte and the image files have identical timecode or frame count. Usually, the effect shot and the matte are trimmed to the required frames, with handles if specified, and have the same duration. The file names should be identical except for a suffix for the mattes. Mattes and images are, however, usually placed in separate folders to simplify the conform process. Clearly, if two files with similar filenames and identical timecode exist in the same folder there is a good chance that the matte and the image will get confused during an auto conform.
 
In the modern DI workflow, the first conform happens before picture lock, which allows the colorist to export precise frame sequences for VFX. The colorist’s first pass focuses on neutralization, matching the adjacent shots in the cut at the time. The colorist may have already started work on the Look and more detailed select parts of the frame, but these should not be present in the export. The ideal export format is EXR, as it can preserve all the available source data. In current practice, VFX facilities prefer to have exported frames in the original camera format along with CDLs that encode the first pass neutralization. This reduces the likelihood of issues that crop up when images have been processed by different packages that may have differing implementations of transforms that should match. Input Transforms are currently a prominent source of variation between software packages. When returning finished VFX for integration into the DI conform master it is important to have a clear policy on handles, color gamut, white point, file naming and format. VFX may come from a variety of facilities around the world, so if each team use a different process they are unlikely to integrate well. Generally, VFX shots returned to the DI conform should retain the naming convention and timecode of the supplied plates with a suffix appended for versions and dates. Technically EXR is again the preferred format, but productions may choose DPX or a Quicktime or MXF wrapper for practical reasons.



\subsection{Unmanaged Output Referred Grading}

The advantage of the output-referred “video” approach is one of process simplicity. However, when grading pre-rendered gamma-encoded imagery, a relatively modest amount of color adjustment can be applied without introducing artifacts. The downside to this approach is that much of the detail in the original imagery is lost when baking-in the view transform, which typically happens before or at the beginning of the grade. Indeed, in some implementations, this baking-in of the view transform may even be internal to the camera, or in animated features, inside the renderer. For example, if a shot was originally overexposed, a sensible color correction is to darken the image. However, it is likely that in an overexposed image large portions of the image will be clipped by the view transform to a constant maximum value, and no possible correction can bring back this lost detail. Additionally, this video style approach is particularly unsuited to producing HDR versions. It is inherently tied to the dynamic range of the display used in the DI, so the HDR and SDR versions must be done as completely separate grades.

\subsection{Color Managed Scene Referred Grading}




An original log film plate (top-left) is loaded into the color grading system. When viewed with a film emulation table (top-right), the appearance is predictive of the final output with a default exposure. If the exposure is lowered using an additive offset in log-space (lower-left), new details in the flame highlights are revealed. When a similar adjustment is applied in an output-referred space (lower-right), this detail is lost.
 Imagery from “Spider-Man” Courtesy of Columbia Pictures. © 2002 Columbia Pictures Industries, Inc. All rights reserved.
Going back to the overexposed example, all of the dynamic range from the original camera capture is preserved when using a scene-referred approach. Thus, when the exposure on log data changes, new details which had not previously been visible through the view transform may be revealed as no clipping has occurred. This allows color grading in log DI to be very high fidelity and most trims do not result in a flat black or flat white in the image.

In the color managed approach, scene-referred imagery is loaded into the machine along with a viewing transform specific to the target reference display to create the final Look. The color grade manipulates the underlying scene-referred representation, but all color judgments are made previewing through the output transform. It is increasingly common to create an approved master for the main delivery format and then to use a trim pass in conjunction with each different output transform required. The trim pass can be an additional grade or metadata.

Camera raw images are the preferred choice of the colorist for the master grade. Failing that a camera-specific integer log encoding is utilized. Camera raw images are typically recognized by the main DI grading systems and facilitate the use of the correct camera transforms.

DPX was commonly used as a log delivery format to DI and both 10 bit and 16 bit integer versions were common-place. For grainy material 10 bit is often sufficient. DPX is still used for film scans and integer sources such as a camera debayering. When passing true scene-referred linear renders directly to the DI grade, 16 bit floating-point EXR is preferable. 16 bit integer linear representations are not recommended.

All DI grading systems process floating-point internally and can happily read EXR files, which have replaced 10 bit DPX as the preferred interchange format. However, the size of EXR and 16 bit DPX is currently costly for storage, processing, and bandwidth, so their use is far from ubiquitous, and some productions still use 10 bit DPX, even though it is less than ideal. 

For viewing, a suitable output transform for the target reference display is applied. The DI grading transform should match the Look Management transform. The color managed workflow should be agreed on and shared with everyone on the project, including those on set and during production to maintain consistency of the images and avoid the need to bake in the transform until the delivery stage. The scene-referred working format is an excellent archival master. The graded scene master can generate multiple deliverables and is the main asset but un ungraded version is often kept too. More information is available in the ACES Digital Source Master specification. Resolution and bit depth of all work should ideally be at least the same as that of the camera original media.

For a simple, robust color managed workflow, ACES as described in 3.1 Academy Color Encoding System (ACES) is ideal. ACES is supported in a wide range of software applications and the standard input and output transforms are reliably consistent. ACES uses EXR files to store scene-referred data that encompasses all of the visible spectrum.

When film was the primary theatrical deliverable, the viewing LUT used to be matched to a specific film lab and print stock, but that is rarely true now. Today theatres and broadcasters expect digital masters and the reference monitor should use the largest color encoding of all the deliverables. For the theatrical exhibition that is usually DCI P3 but for most broadcast deliveries it is still Rec. 709. VOD platforms and games devices are targeting HDR displays, which are currently limited to P3 color gamut with D65 white point, delivered as Rec. 2020. There are several flavors of HDR but they all use either PQ (Perceptual Quantiser) or Hybrid Log Gamma transfer functions. It is quite simple to convert between the two, but since PQ is by far the most common in practice, mastering to a PQ reference monitor, P3 gamut and D65 white point are recommended. All deliverables can come from that. Whatever the output transform is, it is highly recommended that the master is created by color grading the image in a scene-referred color space and making visual decisions based on the appearance after the output transform.

Standard practice is to start with the DCI theatrical master even though it will not always serve as a universal master for other deliverables. Artistic intent and the bulk of the creative decisions are inherent in the theatrical master so the DI grading project with a broader output transform can be re-used as a very efficient starting point for the universal master. Dolby Vision cinema, Eclaircolor and Samsung Onyx masters can only be color graded in a theatre on a screen of the same type as the target cinema. These deliverables are only ever created in their final viewing environment.



Artistically, the DI process can be segmented into a per-shot correction that neutralizes shot to shot color variation, and then a secondary grade that crafts the overall artistic look of the film. With those two adjustments as a base, further work is carried out to manipulate audience experience of the scene and to address any distractions or inconsistencies. As a rule colorists work on sequences, though sometimes with dynamic changes in a shot. Any work that needs to be done frame by frame is usually best done as VFX rather than DI grading. It is common for the DI facility, if an initial grade happens early enough, to communicate to the VFX houses the Looks being used. This often is sent as a CDL, or a 3D LUT, per shot and does not have the viewing transform baked into it. It is also important to communicate the color space that the color grades have been applied in as the CDL or 3D LUT will not produce the intended result without this information. Most often, the DI grading reference display is calibrated to a color space that is the same as the delivery format. 
Key Points
A color managed scene referred grade is recommended
Grades should be applied to a conform of original camera files
Multi layer EXR files are the preferred interchange format for VFX and mattes
Separate matte files can be used if the master image is unchanged
All sequences exchanged in post-production should have fixed frame handles
All sequences exchanged in post-production should retain existing metadata


\subsection{DI Color Grading}

The final DI grade takes place after the conform, but often before picture lock. If possible the colorist starts the grade before VFX are ready and expects to add them to the conformed timeline along with further editorial changes. The grade has three main objectives. First, it must match consecutive scenes seamlessly, even though they may come from different sources or different color gamuts. Second, it is the final stage before delivery and the whole project, not just the images must look the best it can. Whilst the phase is known as DI grading, the DI colorist acts as a crucial QC of the finished master. The DI grade may be the only environment to have large, calibrated displays and might also be the first time that final audio, VFX, titles and composites are seen under critical monitoring conditions. In the ideal world, problems would be sent back to the appropriate team, but as red carpet day looms it is often the colorist who has to fix any remaining problems, even if those problems are sound sync, reflections of the crew in shot or errors in the credits. Thirdly, most important of all, the grade must enhance the concept by communicating emotion, environment and clarity. Like VFX, the best grades are often the least noticeable yet still manage to add depth to the narrative or content. Early on in the DI process, the colorist, editor and VFX departments collaborate and interact. A post-production supervisor might manage the interaction, but as deadlines loom, remaining changes are increasingly left for the colorist. It saves time and helps to keep things running smoothly if files are exchanged in an orderly way.

The project Look is designed to some degree in pre-production. In animation, the look is usually established quite early on. In games, the look evolves as the game develops. In DI grading the Look can be established and communicated through a common transform or show LUT or it might be left until the final DI grade itself. If the colorist is given access to the scene-referred source files and information on their color gamut, a well-managed color pipeline can and should be used through the final mastering stages. Raw camera captures usually contain sufficient metadata to identify their encoding and color space. However, if the files have already been encoded to a more manageable wrapper, ProRes, for example, problems can arise. Colorists often receive files that are either not scene-referred, or the information about their true origin has been lost along the way. When this happens, the colorist can only use experience and good judgment to grade the project and the opportunity to finish through a properly color managed workflow is often lost. If it is absolutely necessary to simplify the DI grade by working from a conformed source in a compressed wrapper format instead of the original source files it is crucial to preserve information about the source and history of  each shot and useful to ensure that all scenes are correctly transformed into the same scene referred working color space. Failure to do so can result in problems and a loss of quality that should have been easily avoided.

When EXR source images or camera raw files are delivered for DI grading the established Look for standard dynamic range is used to begin the initial grade for digital cinema. The source images may be supplemented with additional matte images created by VFX that allow individual characters, or even specific elements like hair or eyes, to be isolated for color grading. Following the completion and approval of the digital cinema grade, a trim pass is performed for theatrical 3D exhibition. Typically this will involve an increase in contrast and saturation to compensate for the lower projection brightness in this environment. Next, a trim pass is performed for theatrical HDR. The standard range look is swapped out for a matching look with extended highlight range. Because theatrical HDR is capable of much darker blacks than standard digital cinema projection, black levels may be further adjusted during the trim pass to maintain appropriate levels of contrast. Rec. 709 for home deliverables is also derived from the digital cinema grade as a trim pass. In this case, a gamma adjustment is made for the change in the viewing environment. Also, the Rec. 709 color gamut is smaller than the digital cinema P3 gamut so some amount of gamut compression may be necessary to preserve image details that would otherwise clip. Lastly, a home HDR grade is performed using the digital grade as a starting point, but substituting an HDR look that targets the much brighter highlight range, color and shadow detail achievable on HDR televisions. The home HDR grade is followed by an analysis to generate HDR metadata. For deliverables that require dynamic metadata, such as Dolby Vision, the HDR master is reviewed by the colorist on different displays and further metadata adjustments recorded.

There is a range of grading systems in use today, and very few of their tools are interchangeable with each other, let alone VFX software. This makes it a challenge for VFX teams to apply reversible transforms or to view accurate simulations of the DI grade without baking it in. The most common methods of previewing VFX are LUTs and CDL. Most color grading systems can export grades as a LUT, however, the limitations of LUTs apply. The two main limitations are that they apply changes to the whole image, with no ability to apply specific changes to selected areas of the image, and that they often clip or produce interpolation errors. See the Appendix 4.4 LUTs and Transforms for a more in-depth discussion. 

Whenever possible, it is best to preview shots in a grading suite or theatrical environment as a confidence check. When there are multiple facilities contributing to different aspects of the show it eventually comes down to getting together in a color suite with a calibrated and controlled environment to ensure that all of the moving pieces do in fact work together. For example, if two VFX plates are cut together and one is slightly warmer, removing red would make a purple suit not match the following shot, so the purple needs to be adjusted separately. Adjusting individual elements through the use of mattes, keys and shapes is called secondary correction. This goes for the sky, skin tones, background and more, even without an overall change in direction of the Look. Often the Look continues to evolve after production and VFX right up to the finishing of the project, which happens in the grading suite.

Key Points

\subsubsection{DI Grading Tools}

Color grading tools have evolved with technology. In the early days of film, adjustments were limited to exposure and color tinting. With color film came color timing with RGB printer lights, which are quite literally filtered lights to alter the color of the print copy. In the video era, analog and then digital tools were developed to optimize film transfers to video via a telecine. The DI grading software of today typically copies or emulates the historical processes, and adds new features from computer graphics, VFX, digital cameras and improved color science. New software grading tools appear slowly, often a long time after the technology that requires them. At the time of writing, there are few special tools for HDR display grades for example. Colorists consequently have become adept at repurposing tools to fit new needs. Grading software has become sophisticated enough that it is rarely possible to recognize the exact tool or software from the result. However, there are significant differences in alternative approaches to the grade. These differences must be considered part of the creative process but a review of the most common techniques can help to understand how a look is crafted and how look management is integral to the process. In short the tools used and the order in which they are used, affect the outcome.

The following section discusses tools and workflows in more detail as the workflow, or more specifically the working color space the tools are used with changes the effect of each tool. Some systems have tools that attempt to achieve the same behavior in each different color space, although that mandates a full color managed approach. Other systems have less sophisticated solutions in which the range of a tool is altered according to the color space and many systems have controls that can be tweaked by the user as needed. However, the most basic commonly used tools are surprisingly agnostic about where they are used, and consequently, they feel and react differently in different workflows. The controls are written with a capital, e.g. Offset, to distinguish them from pure mathematical operators e.g. offset.

For simplicity, an unmanaged color workflow reveals the raw nature of two different grading tools. An unmanaged workflow is a workflow in which the source color space, the timeline color space, and the output color space are all the same and so no transforms are applied by the system. This is still the default for many DI grading systems and many colorists continue to work this way since it is easier, usually requiring no project set up at all. This section, therefore, describes color grading tools as used in the older WYSIWYG approach and is output-referred. This is not the recommended approach but serves to best illustrate the nature of the toolsets and how they are used.


ARRI LogC source image with references. Image © Geoff Boyle All rights reserved.


For the purposes of discussion, the above test image is processed in different ways. The source image has a deliberately wide dynamic range captured with an ARRI camera as ALEXA LogC. The color chart and the two faces at the top of the image are from the camera. Bottom left is a waveform representation of the processing applied to the image and below that a grayscale ramp from which the waveform is derived. The waveform does not reflect the image itself, only the processing applied. Bottom right is a vectorscope display of the color chart which was part of the camera capture. The vectorscope only shows the chromaticity of the chart after processing and is a useful way to compare saturation and hue shifts. The image above is unprocessed so the waveform and ramp show a true linear response. The following examples are not color graded in the true sense of the term, they are examples of the starting point for a DI grade. This type of adjustment to get a reasonable starting point is sometimes called the Technical Grade or a dailies grade. On a real job some color balancing is often needed too, but for this test, neither color balance nor saturation was adjusted. To obtain a realistic comparison the test image was graded using a limited toolset with a waveform as a reference. The grade sets a technical black level and enough contrast to use as much of the available dynamic range as possible, whilst retaining a pleasing result on both pale and dark skin in the same image on a calibrated Rec. 709 display. 

These examples are not exhaustive, there are countless variations possible. The purpose of this exercise is to underline the absolute need for good Look Management. The above image is scene-referred, so the real world examples shown here are intended to demonstrate that using no color management or different color management causes the Look to change significantly. At best, this results in more work to maintain continuity. At worst the intended Look must be compromised or completely re-thought in order to deliver high-quality images. Goo d Look Management is all about consistency.

The three trackballs that make a grading control surface so recognizable are initially mapped to Lift Gamma and Gain (LGG) in most systems. Since these are the first available controls it is not surprising that they are the most commonly used by many colorists. The actual math behind the LGG set can differ between manufacturers, and a few systems offer several versions, but these tools originate from the telecine era of transferring film to video. Since film stocks have different response to light in the shadows, midtonesmid-tones and highlights, colorists needed separate controls for them to manage the film dynamic range in the much more limited video gamut.

The Lift or Black control pivots around white and affects all code values except white. The definition of white is typically code value 1023 in a 10 bit range but again can vary among tools and systems. The tool is ideal for balancing blacks in any color space because the darkest tones are always affected more than the lighter tones, but it is not very selective. A characteristic of this tool is that it pushes the darkest parts of the image, the blacks, harder than any other tone, so when lowering the black point it is prone to causing compressed or clipped shadow detail. In look creation, sometimes this is desirable and sometimes not.

Gain is similar in use to Lift but pivots around the black point or 0. Again different tools and systems can define the black value differently, but if the black point is anything above 0, further conditions need to be defined so that values below black are not affected. When increasing Gain to brighten an image, the brightest values are the first to be clipped or compressed. 

The Gamma control varies between systems far more than lift and gain but is always a simple curve that distributes values towards shadows or highlights. It should normally not affect the absolute black or white level. In many systems, the Gamma control has a fixed range that is output-referred. Consequently, when used on a log source in a scene-referred workflow the code values that become black and white levels at the display are well within the range of the Gamma control and do affect blacks and whites. For this reason, many colorists avoid a simple Gamma control altogether.



ARRI LogC image graded with Lift, Gamma, Gain, and no color management. Image © Geoff Boyle All rights reserved.

In the test image Lift was used to lower the deepest shadows to black on the waveform and Gain was used to bring the brightest details in the flowers to white on the waveform. A small Gamma control adjustment was then applied for a pleasing start point. Without color management, the saturation remains low and there is better hue separation in the warm colors than the cool ones. The result is a quite contrasty, hard looking, low saturated image. The waveform and ramp show clipping, but since none of the values in the test image are in the clipped range that is not a problem. However, it is worth noting that the definition of the white clip point is arbitrarily chosen by eye in this example. Setting the whites to clip in this way is the beginning of crafting a look.

When DI grading was first introduced around 2000, there was some controversy over the use of these video tools in a film to film process. In film timing, there were less sophisticated tools and many felt that in order to retain the essence of the negative to print chemical process, those tools should be more closely emulated. Interestingly, the resulting DI tools, Offset and Contrast, did not exist in any of the video hardware color grading systems, though there were comparable tools in the telecines themselves.

Offset is an additive function applied to all the image code values, so on its own, it does not differentiate any tonal range from another. It is confusingly labeled Offset, Black, Lift (incorrectly), Exposure, Density and probably some other terms according to the system in use. 

In scene-referred linear, a gain operation is typically used to change the color balance and scene exposure. In log space, this roughly corresponds to additive offsets. If a mathematically exact log is used, they are in fact identical, though most manufacturers tweak the log encodings as previously mentioned. Log offset color adjustments per color channel are ubiquitous in motion-picture industry color grading and often referred to as the “primary grade”. Theatrical “fades to black” have a very particular appearance to them, which is a direct consequence of the fade being applied in log space, as viewed through the traditional “S-shaped” film emulation. Fade to blacks applied as gain operations in scene-referred linear have a roughly similar visual appearance, with specular highlights lingering on screen well beyond mid-tones.

The Contrast tool in a workflow that is not color managed is usually an s-curve with an adjustable pivot, applied after Offset. The contrast tool pushes low mid-tones towards black and high mid-tones towards white and approximately simulates the response of a film print stock. It has a less aggressive look than lift and gain and is similar to a color management tone map curve. In color managed systems the output transform is likely to include an s-curve and the Contrast tool might then be a straight line pivoting around an adjustable mid-point so that the two do not conflict. The overall visual result is still an s-curve.


ARRI LogC image graded with Offset / Contrast and no color management. Image © Geoff Boyle All rights reserved.

In the test image, Contrast (S-Curve) was used to adjust the dynamic range to the display and  Offset to bring the range out of clipping. The contrast pivot point was adjusted to maximize the difference between shadows and highlights. Once again this was a purely technical approach based on the image content. Interestingly, this approach produces an even harder look and struggles to get a good representation of the darker skin. A lower contrast setting would work better for both faces but would have left the image looking milky with either lifted blacks or whites below peak. The waveform and ramp show that there is no clipping, which would make it better for a viewing LUT. Chromaticity is only slightly changed from the LGG example. When Offset is used with Contrast, the Contrast control shapes the results.

There is a third type of control found in many systems, that consists of three tools often described as shadows, mid-tones and highlights. These tools are more selective and curve away from a pivot point. The pivot points may be user-defined and the shadow, mid-tone and highlight ranges may or may not overlap, The result is something in between LGG and Offset/Contrast. A single grading system may have just these basic two tool sets or many more variations to choose from. In some systems, the tools are modified slightly by the project settings for color space, and in others, they can be redefined each time they are used. Where there are many variations of the primary tools, the processing order is important and again systems vary. Sometimes the processing order in each layer or node can be altered and sometimes not. When the processing order of tools is fixed, the colorist needs to use more layers/nodes to achieve the same result.

The above tools are all RGB. The 3 channels can be ganged together and used to manage brightness levels. This is called the Master control by colorists. Alternatively, each color channel can be adjusted separately either by a ball or circle or with a knob or slider. Adjusting individual channels allows balancing or neutralizing a color bias but is also used to create a look or style. Three channel RGB tools are commonly referred to as Primaries, not to be confused with RGB color space primaries.

In output-referred grading these primary adjustments are made to bring the image to a viewable state on the reference monitor in lieu of an output transform, whereas in scene-referred grading their result is subsequently processed through the output transform, which alters the effect of the controls. The decisions are based on aesthetics but intended to be neutral. The only definitive information in the image is an unexposed black, the precise values of all other tones and colors are unknown in the absence of Look Management.

The colorist crafts the Look after the initial technical grade using more refined and sophisticated tools. Secondary controls are used in conjunction with external mattes, keys, and shapes to achieve selective changes. The DI grade is built up in layers or nodes, depending on the interface, which can be interconnected and blended together. The secondary controls can be used as part of the Look, or to modify it. For example, if the Look is dark and moody it is common practice to use secondaries to brighten skin tones.

\paragraph{External Mattes}

Sometimes, grading will require external mattes for CG shots, so it is advisable to discuss the need for alpha channel or separate mattes ahead of delivery. The mattes used for grading may not exactly match those used in compositing and can, therefore, incur additional costs.



A sampling of mattes from WreckIt Ralph 2 used to enhance DI grading. 
Image ©Disney. All Rights Reserved.
Animation especially can generate and utilize large numbers of mattes allowing the colorist to isolate individual characters and even specific features such as hair and eyes for fine corrections. It is not unusual to have more than 30 mattes per scene in animated features. Mattes also allow the colorist to separate foreground and background elements, and to rebalance lighting, further improving the continuity of shots even after rendering has been completed. This can save time that would otherwise be spent on multiple render iterations. When the Look has already been established early in animation pre-production and there is a process in place for notes to be recorded during production lighting reviews, the colorist generally starts from a position that is very close to the expectations of the art directors and can focus time on addressing the specific notes that were recorded during reviews.

\subsubsection{DI Grading Workflows}

There are as many workflows for DI grading as there are for VFX. Some start as early as pre-production, some involve live grades during production and some do not assign the finishing colorist until quite late in the project. Collaboration between VFX and the Color department always pays dividends. If the grade begins after VFX, which is not unusual, then plates are likely to be delivered as camera files and need neutralizing. Without a clear direction of the final look, flexibility and latitude are even more critical. If a colorist is on board in time, look creation will have at least begun, and it is reasonable to expect neutralized plates from the Color department. The file format and viewing arrangements can vary according to time, budget and complexity, but scene-referred plates without the look applied are recommended. An ACES EXR file is the most reliable option.  While CDLs are good for providing a general direction for the Look, they are quite limited in scope. A 3D LUT can achieve more selective changes but is less useful in communicating alterations. It is possible to use a CDL and a LUT providing all parties can handle both. There are bespoke and proprietary tools which work well, are more precise and better for communicating change, but again only useful if all parts of the process are compatible. Even with good planning, seeing shots on a big calibrated screen in a controlled environment often reveals things for the first time and adjustments have to be made.  Highlights that distract from the action might need dulling, or an important part of the image might need enhancing to bring more attention to it. These changes might cause a ripple effect and other elements might then need matching back to the new tweaks. Color finishing is fluid and ultimately where all the pieces of a project come together.

In animation, Look Management is usually settled at the start of production rather than at the end of post-production. The Look is much more likely to remain consistent up to and including the mastering stages of the DI grade. The DI Grade for animation is expected to focus on scene-to-scene continuity and smaller adjustments to highlight story points, less so on crafting the full look of a sequence or project.

Look Management should determine the methods by which elements are monitored and quality checked. It is, of course, appropriate to use waveforms, vectorscopes, and histograms to measure scene-referred values, but it is often reassuring to also see something output-referred. Since different approaches for output-referred previews produce very different results, a common method is invaluable. The test image from the previous section Section 3.6.4.1 DI Grading Tools was also processed with different color management solutions to demonstrate the risk of mixed approaches.


ARRI LogC image with ARRI LUT applied and no color correction.  Image © Geoff Boyle All rights reserved.
This first example is the most obvious and in some ways the truest to capture. It uses a LUT supplied by the camera manufacturer to optimize chromaticity and prepare the image for the EOTF of a Rec. 709 display. The ramp shows that the LUT tone maps all possible values without clipping. The result is better chroma separation and saturation than the primary grade examples, but also a flatter image. In fact, this representation is probably the most accurate and would serve as a good starting point for grading. In the DI grade color trims can be applied before the LUT, which keeps the process truly scene-referred. However, it is also common practice in television grades to color grade after the LUT. Grades after the LUT are output-referred and, as discussed previously may limit the ability to recover highlight and shadow detail. Additionally they cannot be used with a different output transform to target other display formats. 


ARRI LogC image with the ACES Rec. 709 Output Transform applied and no color correction.  Image © Geoff Boyle All rights reserved.


This is the result of the test image output from ACES after the correct ARRI Input Transform and the Rec. 709 Output Transform (OT). The result has more punch and slightly more saturation than the ARRI LUT. The S-curve tone map is steeper and shows a different response in the shadows near black. In an ACES project grading is always scene-referred because the OT is managed by project settings rather than as a grading node. This lack of ambiguity is one more reason that ACES is considered the simplest, most reliable form of color management by so many. A cinematographer using ACES to monitor on set can be certain that VFX and DI grading can display the images exactly as seen when shot.


ARRI LogC image with simple color management and primary grading.  Image © Geoff Boyle All rights reserved.

This image is the result of a simple color management system in which the source and working color space are set to LogC ALEXA Wide Gamut and the output is set as 2.4 gamma Rec. 709 and all grading is, therefore, scene-referred with minimal transforms. Since there is no tone mapping applied by the color management it was necessary to use a small Gain adjustment to bring the image out of clipping. This was done by taking the brightest pixels, the highlights on the flower, and bringing them down just out of clipping. The resulting image has the lowest contrast of all the tests and shows details even in the dark skin. The shadows on the pale skin are so light that the look is rather unnatural. The chroma separation and saturation are well defined and clean. This would make a decent starting point for a commercial or a Look for beauty products but because the color management alone does not tone map the entire dynamic range to that of the display. Exposures higher than the test image would appear clipped on a Rec. 709 display, even though the source data might not be clipped.

The results from using different color tools or color management would cause for confusing conversations at best, but the greater concern is that finished scenes could be hard to match in the final conform because although they were created scene-referred, they were seen differently when they were created. Even the decision to apply tweaks to an otherwise standard color managed pipeline is a material fact that can considerably affect the time taken to achieve the DI grade and can in extreme cases necessitate a late change to the established look in order to hide discrepancies that were not seen in the creation stages. 

The above examples are not in any way exhaustive but clearly demonstrate the importance of Look Management. All of the examples are valid viewing transforms without an intentional creative Look applied. Any of these methods might be chosen as the starting point for crafting a look. Considerations in making that choice range from repeatability across multiple software, tone mapping capabilities to different display formats, and the ability to display good saturation and contrast, to the best aesthetic choice for managing the two skin tones in the same frame or preserving the hair color most accurately. These comparisons are kept simple for demonstration purposes. Real world examples could include considerably more effort to include an established Look on monitored previews.
XXX

