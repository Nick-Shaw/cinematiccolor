\section{Finishing}

It is common in DI to create masters for multiple output devices. On major motion-pictures, masters for digital projection, film release, home theater in both HD and standard def, a mix of 2D and 3D, and most recently in wide gamut HDR are expected. The general approach is to identify one output as the gold standard and to spend the majority of the artistic time grading the images to look perfect on that device. For cinema, the theatrical digital projection is most often the appropriate master to treat as the reference. The director, producers, and other creatives will be present at this process. Once the main color grade is complete, additional masters are handled as trim passes atop the main output. Trim passes are most often implemented as additional color layers or nodes added after the master grade, and only utilize minor adjustments such as contrast, brightness, and saturation. The display transform baked into each master is device-specific; a different view transform tailored to each output device is a necessity. Where it is for conversion to a smaller colour space, and involves no change of dynamic range, it may be possible to perform the trim pass in an output-referred space. Creatives do not need to be present during the creation of trim passes, as the colorist is already familiar with the final look of the material and has a known reference to use as a color appearance target.
To support deliverables for standard displays as well as next-generation displays, a workflow utilizing the EXR image format can be used. A creative look and tone mapping are followed by a display transform for viewing. For HDR deliverables, the standard look has a complimentary HDR look that can be swapped at any time for review or grading on an HDR display. This allows the production team to review their work on both conventional displays and in HDR to ensure the range of image values is technically accurate and creatively consistent in both formats.
There are situations where some of the deliverables have a different artistic intent to the master, such as when a UHD HDR deliverable created after sign off on an SDR master. Because this deliverable significantly increases the range of colors beyond what was available during the initial digital cinema grade, a decision can be made to either create the HDR deliverable through a trim that stretches the original grade into the larger color volume of HDR or adjustments can be made in the grade itself to more fully utilize the additional information inherent in the source images. If a LUT has been used to apply the creative look for SDR, it may also be swapped out to an HDR look during this stage. The HDR format can reveal compromises in the source material that were not visible in the SDR master and require other corrections needed specifically for HDR. This needs to be considered to ensure enough time is allocated for HDR finishing.

The motion-picture viewing environment greatly impacts color appearance.  Theatrical viewing environments typically have dark surround and a relatively low screen luminance (14 fl, 48 cd/m2 is specified by DCI). In a desktop/office setting, an average surround is likely. Some common calibration targets demonstrate the variety of environment/ display combinations. 

	Display			Environment		Screen luminance	Gamma
Theatrical digital projection 	 dark surround		14 fl = 48 cd/m2  	2.6
Home digital projection 	 dark surround		16 fl = 55 cd/m2    	2.4
Reference TV monitor 	 dim surround		29 fl = 100 cd/m2 	2.4
Home TV monitor 	 	 average surround	50 fl = 170 cd/m2 	2.2
HDR10 monitor		 average surround	292 fl = 1000 cd/m2 	PQ

Thus, the same image displayed in each environment will appear very different. Specifically, the perceived image contrast and colorfulness will vary. Color appearance models may be used to attempt to correct for these settings, but trim passes with a human in the loop typically yield far higher fidelity results.

When mastering for stereoscopic 3D, an additional 3D specific trim pass is necessary. Due to the additional optics necessary to create the stereo image pair including additional filters in the projector and glasses on the audience, the projected stereo image is usually far lower luminance than traditional 2D theatrical projection. Early projection systems were limited to levels as low as 3.5 fl. As lower luminance images tend to appear less colorful, the trim passes for stereo masters typically boost both the saturation and contrast relative to the 2D master. However, this boost can impose a cost on image detail, for example clipping highlights. Also, the increased saturation and contrast will not be suitable for projection at significantly higher luminance. As projection technology has matured to support higher luminance projection this has led to the creation of multiple 3D deliverables. One way of managing this is to create a trim with boosted color for standard 3D venues, and a full brightness 3D grade without any color boost for venues that are capable of projecting at or close to 14 fl.

\subsection{Multiple deliverables}

There are two different approaches to multiple format deliveries; one is that all deliverables should look the same or at least as close as possible. Clearly, making all decisions in the smallest color space (Rec. 709) makes it straightforward to deliver in all other formats. Alternatively, making decisions in the largest available color space will mean that other deliverables may lack some of the color or brightness of the master, but the overall quality is likely to be better. For example, the Dolby Vision workflow gives the DI colorist responsibility for all deliverables using scene by scene metadata, that effectively extends control over the image, all the way to the playback device. Similar decisions have to be made for resolution and aspect ratio.

\subsubsection{Resolution}

The common resolutions for home deliverables are SD for DVD releases, HD for broadcast, Blu-ray and streaming, and UHD for newer channels including VOD and Blu-ray. Theatrical releases may have different aspect ratios but are also spread approximately between 2k and 4k resolutions. It is sensible for most production and post-production to target a 4k or better finishing pipeline. Editing can usually use lower resolution proxies, but full resolution files should be checked before making decisions. For wholly computer generated sources such as animated features there is a debate about whether 4k resolutions are needed.

