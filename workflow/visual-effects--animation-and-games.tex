\section{Visual Effects, Animation and Games}

\subsection{Visual Effects}


Captured camera log plate (top), ACEScg workign space image (middle) and Rec. 709 final image (bottom). Shot form “Avengers: Infinity War”
© 2018 MARVEL
The traditional visual effects color pipeline is based around the core principle of doing no harm to the plate, to allow for the seamless intercutting of VFX and non-VFX shots. Plate photography is brought into the pipeline in whatever color space is delivered, typically the camera log, and then converted to scene-referred linear values using an invertible transform. Sometimes the camera’s encoding gamut is kept and only a pure 1D camera to linear transform is used. With a more modern approach, the linearization is accompanied by a matrix transform to convert to a wide-gamut working space like ACEScg. A working space gamut separate from the camera encoding gamut is especially useful if multiple different cameras, each with its own encoding gamut, are used during shooting. For visualization, a 3D LUT is used which emulates the eventual output transform. This was traditionally based on a print film emulation or a similar “S-shaped” curve, but is now more commonly created entirely in the grading suite. During the visual effects process, the output transform is never baked into the imagery except for intermediate deliveries such as editorial output or rough preview screenings. These deliveries are used to enable other parts of production, but are not the “final” deliverable. The delivery to the colorist should be the highest-fidelity representation of the original photography. Getting to this highest-fidelity representation typically involves going back to the camera encoding gamut and camera log space used when delivering plates, or rendering scene-referred linear to EXR files in a specified gamut.

\subsection{Animation}




A series of the elements used for a shot from “The Incredibles 2”. The three images represent the linear, Rec. 709 working space image (top) and the default look for a Rec. 709 display (middle) and the final show look for a Rec. 709 display.
 Images © Disney/Pixar
In animated features, all imagery is generated synthetically. Plate invertibility is a non-issue as there are no camera plates to match. The lack of real-world elements to match typically means that far more more latitude is allowed in the selection of viewing transforms. Despite such freedoms, it is often preferable to carry over those aspects of the traditional film workflow which are beneficial even in a fully-CG context. Think of an animated feature as creating a “virtual movie-set”, complete with a virtual camera, lighting and target display. Of course, while these virtual color transforms can be inspired by real-world counterparts, it is generally preferable to use idealizing simplifications which do not delve into the peculiarities inherent to color in physical imaging workflows.
Physically-inspired animated feature color pipelines require selection of a working-space gamut and a visualization, look or output, transform. The working-space gamut is chosen based on the deliverable needs of the project as well as authoring constraints of the production. If the show needs to deliver finals covering Rec. 2020, Rec. 709 and DCI-P3, it is important to choose a color space and authoring workflow that will cover these gamuts. The gamut of desktop monitors is a limiting factor on the creative utilization of wide gamut colors. Ideally, they should be matched to the primary wide gamut deliverable. Multiple manufacturers offer monitors that support P3, making this a viable target gamut for the desktop. Care should be taken that the reference monitor is calibrated to the correct white point and EOTF as well as primaries.
Like visual effects, the output transform for an animated feature was traditionally inspired by characteristics of film print stock but is now more commonly a product of the lighting supervisors, art directors and color scientists. It has diverged from film print stock characteristics due in part to the desire for bright saturated colors in animation, but also because digital cinema has become the primary market. It is now more common to match film prints to the digital cinema master than to match the digital cinema master to a film print. Like in visual effects, the output transform is only baked into intermediate deliveries such as editorial output or rough preview screenings. The delivery to the colorist should be the highest-fidelity representation of the rendered imagery. As floating-point images have gained more widespread support within software and hardware used outside of animation and systems like ACES become more widely used, it is now common to deliver floating-point EXR files directly from the CG pipeline to DI grading. Establishing a look for the output transform early in the animation process makes it the primary driver of the look of the final image. Grading for animations tends to have a lighter effect on the Look of the image than for visual effects, focusing on continuity and fine-tuning of the overall project.
Animated-feature color workflows have historically relied upon a single-step viewing transform, which directly converts from linear to the display, often using a gamma transform. However, physically inspired animated-feature color workflows (which separate the negative vs. print transform) are becoming increasingly preferred due to the ease of working with HDR rendering spaces and the robust hand-off to DI.
With the advent of multiple deliverables, often including HDR, working in a scene-referred space is particularly beneficial. The same scene-referred rendering can be passed through different display transforms, with only a trim pass in the final grade required to produce HDR and SDR deliverables.

\subsection{Games}




A series of the elements used for a shot from Battlefield V. The three images represent the linear, Rec. 709 working space image (top), the ungraded image with a generic Rec. 709 output transform (middle) and the final show look for a Rec. 709 display.
Imagery from Battlefield V courtesy of Electronic Arts Inc, © 2018 Electronic Arts Inc. All rights reserved.
Games and real-time projects are very much like animations in that every aspect of the scene is simulated in CG: the virtual lights, cameras, set and characters, without being constrained to match a live-action plate. Developments in color processing in real time rendering have largely aligned with those in the visual effects and animation space, finding that approaches that maintain physical plausibility by computing lighting and material reflectance in scene-referred linear wide-gamut spaces and that separate the lighting and material reflectance from the display pipeline provide for the most realistic results and the most flexibility in targeting many types of displays.

Games have vastly reduced frame processing time compared to non-real-time CG, due to their frames being entirely generated in real-time, i.e. 30, 60, 90 frames per second or higher. As such they rarely have the luxury of working natively with 16 bit floating point data for textures, and instead operate primarily using heavily compressed hardware-accelerated texture formats which trade quality for a dramatic reduction in memory footprint and increase in performance. See section 3.4.4.1 Textures in games for more on this topic.

Another important performance, quality tradeoff to note is related to runtime floating point formats. When alpha channels are not needed and performance and/or memory footprint are more important than precision, games sometimes prefer to use floating point formats smaller than 16 bit. Colloquially known as “small float” or “mini float” formats, these floating point formats have the same 5 bit exponent size as 16 bit floats. They thus have very similar positive range to 16 bit floats, but they have a reduced size mantissa so precision is reduced. Most importantly these formats also do not have a sign bit, so do not support negative numbers and cannot store out-of-gamut colors. See “Small floats” in Appendix 4.11.

Much like animation, physically-inspired games color pipelines require selection of a working-space gamut and a visualization, or output, transform. The working-space gamut is chosen based on the deliverable needs of the game as well as authoring constraints of the production pipeline. If a game will need to render on devices covering Rec. 2020, Rec. 709 and P3, it will be important to choose a gamut and authoring workflow that will cover these display gamuts. At the time of writing, the majority of games are produced and rendered in the Rec. 709 / sRGB gamut, primarily for reasons of convenience including use of legacy assets and performance. Although some games do natively render in a wider gamut and it is expected for games to rapidly follow Animation into native support for wider color gamuts.




When it comes to gamut handling there does not seem to be a standardised approach in games. Many different options exist, each with their own tradeoffs, and it is likely different games will take different approaches depending on how this best suits their situation. Several options are highlighted below

The engine runs natively in the Rec. 709 gamut and all color data, such as textures, movies, and dynamically animated colors, are provided in this same gamut. Output may be in a wider gamut, but unless post processing also operates in a wider gamut then no wide gamut colors will be available.
The engine runs natively in the Rec. 709 gamut and the majority of color data, such as textures, movies and dynamically animated colors, are provided in this same gamut. The renderer uses signed floating point formats internally and employs the “scRGB” gamut which uses the same color primaries as Rec. 709 but supports wider gamuts through the use of negative numbers. Selected color data may be authored in a wider gamut and encoded in scRGB, requiring use of signed floating point texture formats which usually carry memory and/or performance overheads. See 3.4.4.1 Textures in games for more detail. This option can be appealing if the vast majority of color data is not wide gamut, only a few selected wide gamut assets exist, the performance and memory overheads are acceptable, and no obvious rendering artefacts are visible. See 3.4.6.2 The Rendering Gamut Impact for more detail.
The engine runs natively in a wider gamut, say P3 or Rec. 2020, and the engine employs color management to propagate the gamut of all color assets to the renderer. The renderer dynamically transforms all color data, such as textures, movies and dynamically animated colors, into the native gamut of the renderer at the cost of performance overhead in the renderer. Usually only gamut expansion is supported, as gamut reduction is more complex and expensive. A side benefit of this runtime transformation is that range compression/expansion (scale/bias) can be applied for free, allowing some artefacts associated with compressed textures to be minimised. Small float formats are supported. This option can be appealing if games can accommodate the additional rendering cost and complexity of managing the gamut mapping in the renderer.
The engine runs natively in a wider gamut, say P3 or Rec. 2020, and the engine employs color management to transform the gamut of all color assets into the renderer’s native gamut offline as a preprocessing step. This comes at the cost of a potential loss of quality due to quantisation and texture compression. See 3.4.4.1 Textures in games for more detail. The renderer presumes all colors are in the working gamut. Small float formats are supported. This option can be appealing if games cannot afford additional runtime performance overhead or renderer complexity but can accommodate some quality loss associated with storing wide gamut textures in low bit depth compressed formats, or are willing to pay the additional cost to disable texture compression and/or use less compressed formats in the case that visual artefacts are seen.

These are just a few examples of how gamut can be handled; it is likely that many more options exist and are in use. In any of the cases where wide gamut is supported, the necessary gamut reduction must be undertaken if outputting to a narrow gamut display. What is important to note is the need for color management of the source assets, and that the fundamental principles still align with the other CG approaches in this document regardless of how the renderer handles them.

Games also have a more varied approach to output transforms. Filmic Tonemapping by Duiker (2006) is a widely used approach that started with LUTs based on film print stock measurements but then gave way to analytical approximations, Hable (2010), to the original curve that form their own family of curves. Many games projects use the Reinhard (2005) algorithm, Drago (2003) Adaptive Logarithmic mapping and still other more ad-hoc approaches have been developed to suit each game. In recent releases, Unreal and Unity have both adopted elements of the ACES working spaces and output transforms. Games engines typically combine more limited output transforms with extensive, dynamic color correction controls, typically including an ability to use 3D LUTs.

As the range of displays widens to include larger variations in dynamic range and gamut, separating the working space gamut from the display gamut and the color correction from the output transform becomes increasingly important. One advantage games have over the other CG productions is the entirely real-time nature of their rendering. Games can dynamically adjust their display mapping to best suit the capability of the display and can rely less on fixed mastering levels like the current set of ACES Output Transforms. This can help achieve the best possible quality for each display, but also carries with it gameplay implications. It is important for both competitive and social gaming that gameplay-essential luminance ranges be visible on all displays, to ensure games remain playable and no gamer is at a disadvantage if they are playing on a less capable display. This dynamic display mapping requires knowledge of the display characteristics, which is not always easy to ascertain.


\subsection{Texture Painting}





Theses color maps represent the color modulation of the diffuse, specular and reflection components of a surface shader and the beauty render.
© 2018 MARVEL
Formerly a majority of 2D texture painting applications worked most intuitively in output-referred color spaces, where the painted image was directly displayed on the screen. However, the texture authoring landscape in the VFX industry has dramatically changed with the commercial release of Mari (2010) and other 3D paint packages like Substance Designer (2010) and Substance Painter (2014). These 3D painting packages adopt a scene-referred rendering workflow where artists are able to paint plausible reflectance values while reviewing their work in a viewport that approximates the shading and lighting of the final render and includes the in-house or client output transform. Because 3D viewports are driven by physically-based real-time renderers that strive to present a close approximation of the asset appearance as computed by the offline renderers, artists are able to paint much more in context.

 Examples of the effect of varying material parameters. Each parameter is varied across the row from zero to one with the other parameters held constant. - Physically-Based Shading at Disney, Burley, 2012

One point worth discussing is the somewhat muddied question of whether a texture is scene-referred or output-referred, or if scene-referred, what that means. For diffuse textures, it is relatively easy to say what it represents at any point in the lighting and rendering pipeline: diffuse reflectance values specified in the working gamut used for asset development. For textures that represent specular roughness, normal distribution, sub-surface scattering mean free path or other more technical aspects of a surface’s response to lighting, the texture may represent a linear value, a gamma-encoded value, a log value or values from another bespoke measurement space. The connection between these values and the working gamut for asset development is also less clear. The values in those textures are best thought of as being “parameter-referred” as they only have meaning for the parameterization of the surface shader being used. The choice of which parameters to expose for a given material and which measures and space to use for them is a subject of much debate. See Burley (2012) for an interesting discussion of the considerations involved in parameterizing surface shaders. Given the dependence on the surface shader parameterization, painting textures in a 3D paint package with a physically plausible shader that shares the parameterization of the production renderer becomes much more important. Painting textures in a 2D application for parameters that aren’t easily mapped to a display can be an exercise in frustration, often leading to a guessing game as to what effect a change to the texture will have on the look of a 3D asset. Regardless of the approach chosen, one must distinguish painted color maps and data maps: bump maps, normal maps, iso maps, control maps, etc, which should not be processed colorimetrically.






A shot from “Rogue One: A Star Wars Story” including an X Wing model (top) with the painted diffuse (bottom, left), specular intensity (bottom, middle) and specular roughness (bottom, right) maps.
Texture reference might be sourced from the internet or other sources of output-referred imagery such as the JPG/TIFF output from a digital camera. In this scenario, a common approach is to utilize an inverse tone rendering, to convert output-referred imagery to a hypothetical scene-referred space. Conceptually, the texture artist is painting the tone rendered appearance of the texture, not the texture itself. There are a few challenges with this conversion from output-referred to scene-referred space. First, the tone rendering may not be invertible, particularly when filmic 3D LUTs are used for image preview. Second, traditional s-shaped tone renderings have very horizontal portions of the curve, which when inverted result in very steeply sloped transfer functions. This steep contrast has the effect of amplifying small changes in input code values. For example, a steep inverse could result in the situation where a single code value change in a painted texture could correspond to a delta of a stop or more of scene-referred linear light. This sensitivity is very challenging for artists to work with.

For text and logos, the goal is often to convert to scene-referred linear through the inverse of the project Look, specifically so that when the text or logos are integrated with other imagery and run through the forward version of the project Look, the original text and logo colors are returned. Thus, the overall effect is a no-op on the logo and text.

A common solution to these issues is to be a bit more modest in the goal of perfectly inverting the display transform. Simplifying the problem, we can instead aim to create an approximate 1D inverse, tuned to be well behaved both in terms of color performance and dynamic range. Of course, as a consequence of this simplification, the texture painting is not truly WYSIWYG. Thus, a residual difference visualization 3D LUT is crafted for accurate color preview in the texture painting tool.

Another axis of variation in texturing is when to perform the conversion to scene-referred linear. The suggested approach is to linearize prior to mipmap texture generation. The primary advantage of this approach is that scene-referred linear energy is preserved through all the mipmap levels, allowing for the highest fidelity sampling and fewest color space related artifacts. Further, disallowing color space conversions at shading time prevents shaders from using undesirable, non-linear, color math. The disadvantage is that the storage requirements for linearized data are potentially increased, i.e., even if a texture is painted at 8 bits of precision in an output-referred space, increased bit-depths are required after conversion to scene-referred linear. When dealing with 16 bit painted textures, this is less of a concern as the delta in file size is smaller.

It is common for facilities to have an Onset Capture department whose responsibility is to acquire the texture references for a show. They are often processed with a flavor of DCRaw, Libraw or a dedicated in-house tool that output the texture reference data as scene-referred linear light values encoded with the facility working color space inside an EXR file. In a controlled environment, the use of cross-polarizing filters and polarised lighting during acquisition further allows for increased reference texture fidelity as the specular response of the sample is attenuated producing a more useful representation of its reflectance. See 3.3.5 On-Set Reference Capture for more detail.

\subsubsection{Textures in Games}
As mentioned in section 3.4.3, games are often biased heavily towards the performance end of the performance, quality tradeoff. Games make much use of heavily compressed hardware-accelerated texture formats which trade quality for a dramatic reduction in memory footprint and increase in performance.
Compressed textures are not required, but are often used. In addition, there is not always a one to one correlation between the storage format and the runtime format; some games transcode from one format to another to ensure minimal storage or loading overhead while still being able to leverage hardware texture compression in the renderer.
By far the most common formats are 8 bit and unsigned (storing positive-only values) although some are 16 bit. 16 bit formats are larger and often slower, so tend to be used sparingly but are still useful especially those that are floating point with support for a sign bit. It is usual for textures to be authored at high quality, but resized and compressed in a conditioning pipeline before they are stored on disk or streamed.
HDR formats are supported by hardware in various guises, including several variants of 16 bit floating point: uncompressed, small float formats with no sign bit, compressed with no sign bit and compressed with a sign bit, and compressed shared exponent formats like RGBE also with no sign bit.
Since texture compression is lossy and involves bit depth reduction, artefacts including quantisation can be seen. In this case it can be beneficial to scale/bias SDR textures to use the full 0-1 range before compression, propagate this scale/bias to the game engine and remove it at runtime. This can minimise artefacts for a small per-texture runtime cost.
See “Block Compression” in Appendix 4.10 for more information on compressed textures.
Another GPU hardware feature used extensively by games is support for free OETF/EOTF application. Since scene-referred linear rendering is preferred, but textures are usually stored perceptually encoded, GPUs feature hardware to apply a free EOTF to each texel fetched, linearizing these values before filtering and before they are used as inputs to the renderer. In addition GPUs can apply a free OETF to a value before writing it to memory if this is required. Unfortunately, the only transfer function supported is the sRGB. See 4.1.3 sRGB for the definition of the sRGB transfer function. At time of writing there is no hardware support for any other transfer function, including PQ. Even with that support, it is important to note that only the transfer function part of the sRGB specification is provided; there is no gamut handling in hardware.

\subsection{Matte Painting}

Matte painters and paint packages work in as many different ways as there are facilities and software packages. In this example, the painter works in an output-referred color space. As renders are natively in scene-referred linear, a mapping from the matte painting space to the rendering working space is required, regardless of which approach is chosen.
It is often useful to consider matte painting and texture painting as two different color workflows. In matte painting, it is very common for the artist to want to paint values which will eventually utilize the full scene-referred linear dynamic range, even when working in an output-referred space. Thus, an inversion which allows for the creation of bright specular values is preferable. In such cases, it is often convenient to provide a visual preview both at a normal exposure, as well as a few stops darker, to let the artist have a good feel for the dynamic range in their plate. Applying an inversion in this context has many of the potential downsides described for texture painting, namely that an exact inverse isn’t guaranteed to exist for all color and that the inverse of a tone curve with strong highlight compression may lead to excessively bright values when an output-referred image is brought back into the working space.
To avoid those issues, some workflows elect to perform matte painting on scene-referred values. The scene-referred values are typically encoded in a non-linear fashion for compatibility with integer based tools and to utilize the largest dynamic range possible. One option is to use a logarithmic encoding and to provide a viewing transform that incorporates the look for the show. This will be familiar to artists who have experience with films scans, but it also changes the feel of paint tools that are normally used in a display-referred context. Another option is to use a hybrid log-gamma encoding. This hybrid encoding uses a curve which is gamma-shaped for the standard range of image values and transitions into a logarithmic shape for bright values like highlights. The gamma portion of the curve allows tools that are built for a display-referred working space to continue to work as expected, including blending operations which are designed for gamma-encoded material. The logarithmic shape for highlights allows details to be preserved throughout the process. Artists can paint directly on the matte image without any additional viewing transform. A viewing transform that incorporates the look for the show is still useful, to see how the painted values will appear in their final presentation. These viewing transform can be applied using an ICC profile or a 3D LUT as supported by the paint application.

\subsection{Lighting, Shading, and Rendering}

The stages of rendering, lighting, and shading most closely approximate physical realism when performed in a color space that is scene-referred linear, high-dynamic range and wide-gamut. Ideally, no color space conversions should be required during the rendering process, as all input assets such as textures, plate re-projections, and skydomes, can be linearized and gamut mapped beforehand.

Image viewing of scene-referred linear data is typically handled by converting to the color space being delivered to digital intermediate, often a log encoding of the color space, and then applying the view transform suitable for the specified display. For convenience, these transforms are typically baked into a single 3D LUT, though care must be taken to assure the LUT has suitable fidelity over an appropriate HDR domain. A simple cube 3D LUT is not suitable for use with linear data. A 1D shaper LUT, or mathematical transform needs to be applied prior to the cube to give the mesh points a more appropriate spacing. See Appendix 4.4 for more detail on LUTs.



A rendered image with no output transform applied (top), an image with a generic Rec. 709 output transform (middle) and the show’s final look (bottom).
Images from “Moana” Courtesy of Walt Disney Animation Studios. Images © Disney. All rights reserved.

In this raw visualization of a high-dynamic range, scene-referred linear render, values from the source image greater than 1.0 are clipped when sent to the display. Using a naive gamma 2.2 visualization to map scene-referred linear to output-referred values results in an image with low apparent contrast and clipped highlights. Observe clipping in the wave and eye. Using an S-shaped tone curve applied in a log space to visualize the scene-referred linear render yields a pleasing appearance of contrast, with well-balanced highlight and shadow details.

\subsubsection{Scene-Referred Linear is Preferred for Lighting}

A render is a simulation of a real scene. In the real world, there is a linear relationship between the incoming and outgoing energy from a surface, i.e. the incoming light that is scattered or absorbed. To simulate that interaction properly, working with units that refer to the physical units of energy in the world and applying operations that maintain the linearity of these relationships makes the simulation more straightforward.

First, the render itself benefits. Physically plausible light transport renderer mechanisms such as global illumination yield natural results when given scenes with linear high-dynamic range data. Physically-based specular models combined with area lights produce physically plausible results with high-dynamic range data. Rendering in scene-referred linear also allows lights and material components to be re-balanced post-render, with results that track identically as if the original render had been tweaked. Storing rendered images in a floating-point container is most common to preserve maximum image fidelity; the EXR format is most widely used in VFX and animation and increasingly in games and for interchange in post-production. See 4.9.2 EXR for a more in-depth description of OpenEXR.

Figure 1 from ‘OpenEXR Color Management’ by Kainz (2004), ILM advocating for the use of scene-referred linear values in lighting, rendering and compositing. Many of the devices and acronyms have changed, but the core concepts remain valid.

As with preserving the dynamic range of the camera into DI Grading, rendering CGI in a high-dynamic range scene-referred linear space makes it simpler to produce HDR and SDR versions of the film. Rather than baking in tone and gamut mapping applicable only to one display format, different output transforms can be applied to the same scene-referred image data to view it on HDR or SDR displays.
One issue to watch out for with high-dynamic range renders is render noise. When scene textures like skydomes contain very bright areas, like the sun or other compact light sources, i.e. areas that substantially contribute to the scene illumination with relative luminance values well above 1.0, care must be taken in sampling to avoid noise. Rendering techniques such as multi-importance sampling (MIS) are useful to mitigate such issues. Even still, it is common to paint out very compact or very bright light sources, such as the sun, from skydomes, and then to add them back into the scene as native renderer lights to allow for both lower-noise sampling and often greater artistic control.

Light shaders also benefit from working with scene-referred linear, specifically in the area of light falloff. In the distant past, the default mode of working with lights in computer-graphics was to not use falloff. However, when combined with physically-based shading models, using an inverse-square law light falloff behaves naturally as the relative size of a light diminishes with the square of the distance as a light recedes from a surface, or vice versa. Note that this is a simplification of the full solid angle-based equation for the projected area of a surface or light and that this can be problematic in particular when scenes are modeled with large units, say 1 unit is 1 meter, and it becomes common for objects to be less than one unit from lights. In this case, the distance value less than 1 drives the results of the inverse-square law falloff term to be considerably greater than 1, no longer resembling a falloff. If one tries to shoehorn realistic lighting falloff models into output-referred rendering, the non-linearity of output-referred spaces make it’s very difficult to avoid clipping and generally implausible results. On the downside, one consequence of working with natural light falloff is that it’s often required to have very high light intensity values. It is therefore common to express light intensity in user interfaces in terms of “exposure” or “stops,” as it’s much friendlier to the artist to present an interface value of ”+20 stops,” compared to an RGB value of “1048576.0”.

Anti-aliasing operations also benefit from using scene-referred linear values, though one must be more careful with renderer reconstruction filters. Negatively lobed filters such as Catmull-Rom have an increased tendency to exhibit ringing artifacts due to the extra dynamic range. This is a particular problem on elements lit with very bright “rim lights,” as this creates bright specular highlights directly in contact with edges. The core challenge here is the same as when handling very bright, very small sections of imagery in compositing. There are a number of common approaches to working around such filtering issues. First, for operations that shows ringing artifacts, switching the filter to a box or gaussian filter will remove the artifact, at the expense of softening the image slightly. If the issue only happens in a few places, this may be a reasonable approach. Second, very bright specular samples can be rolled-off such that these samples do not contribute such large amounts of energy. However, this throws out much of the visually significant appearance which adds so much to the realism. Another approach is that the extra energy can be spread amongst neighboring pixels such that the specular hits show an effect analogous to camera flare. Both of these effects can be implemented in compositing.

\subsubsection{The Rendering Gamut Impact}

A shot from “Rogue One: A Star Wars Story” with a green beam of light that is far outside of the Rec. 709 gamut.

One decision that has taken on more prominence recently is the choice of gamut to use for rendering. It is important to acknowledge that the rendering gamut affects the way computations are performed, especially indirect lighting as shown by Agland (2014). Rendering an image using sRGB textures and then converting it to ACEScg will yield a different image than having rendered with ACEScg textures in the first place.

Mathematical operations are dependent on the chosen basis vectors. In the space of color, the basis vectors are defined by the choice of RGB color space primaries. The same operations performed in different RGB color spaces will yield different tristimulus values once converted back to CIE XYZ color space. For example multiplication, division and power operations are dependent on the RGB color space primaries while addition and subtraction are not. Quoting Rick Sayre from Pixar:

“The RGB basis vectors typically become non-orthogonal when transformed to XYZ, and definitely so in this case. Thus there should be no surprise that component-wise multiply does not yield a proper transform between two non-orthogonal spaces.”


Illustration of the effect of multiplying various colors by themselves into different RGB color spaces: the resulting colors are different. The various samples are generated as follows: 3 random sRGB color space values are picked and converted to the three studied RGB color spaces, they are exponentiated, converted back to sRGB color space, plotted in the CIE 1931 Chromaticity Diagram on the left and displayed as swatches on the right.
This introduces the concept that some RGB color spaces are more suitable than others when it comes to 3D content generation and computer graphics imagery in general. This problem is typically solved by using spectral rendering or leveraging a spectral renderer as the ground truth base against which the RGB color space is selected.

Renders of the same scene using Rec. 709 primaries (first row), 47 spectral bins (second row), Rec. 2020 primaries (third row), spectral minus Rec. 709 primaries render residuals (fourth row), spectral minus Rec. 2020 primaries render residuals (fifth row). The last row shows composite images assembled with three vertical stripes of respectively the Rec. 709 primaries, spectral and, Rec. 2020 primaries renders.

Tests and research conducted by Ward and Eydelberg-Vileshin (2002), Langlands and Mansencal (2014) and Mansencal (2014) showed that gamuts with primaries closest to the spectral locus, i.e. spectrally sharp primaries, tend to minimize the errors compared to spectral ground truth renders.
Usage of gamuts such as Sharp RGB, Rec. 2020 or ACEScg is often beneficial for this reason but it is contextual and sometimes, Rec. 709 can perform slightly better. In the previous image, direct illumination tends to match between the renders. Areas that show the effect of multiple light bounces, i.e., the ceiling, in the Rec. 709 and Rec. 2020 primaries renders tend to exhibit increased saturation, especially in the Rec. 709 primaries render or slight loss of energy, especially in the Rec. 2020 render. Excluding outliers, e.g., the visible light source, the RMSE with the spectral render are 0.0083 and 0.0116 for respectively the Rec. 2020 primaries and Rec. 709 primaries renders.

As of the writing, these is no first principal or formal mathematical proof that one particular working space is best for computer graphics. As demonstrated, any choice of gamut will introduce error relative to ground truth spectral renders though wide gamut spaces such as Rec. 2020 or ACEScg tend to match spectral ground truth renders most closely. The working-space gamut is typically chosen based on the deliverable needs of the project as well as authoring constraints of the production. If the show needs to deliver finals covering Rec. 2020, Rec. 709 and DCI-P3, it is important to choose a color space and authoring workflow that will cover these gamuts. See 3.4.1 Visual Effects, 3.4.2 Animation and 3.4.3 Games for considerations specific to different types of projects.


