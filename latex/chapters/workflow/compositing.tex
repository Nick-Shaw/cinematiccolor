\section{Compositing}
\label{sec:compositing}




A shot from “Rogue One: A Star Wars Story”, starting with a CG background plate (top), adding in a live-action foreground element (middle) and the final composite image (bottom). All three images have an Output Transform for a Rec. 709 display.
© and ™ Lucasfilm Ltd. All Rights Reserved

Compositing is the process where live-action and CG elements are manipulated and merged. Image processing in scene-referred linear and logarithmic encoding spaces are both useful, though scene-referred linear is the default. As in lighting, image display typically leverages viewing transforms that emulate the final output. Examples of commercially available compositing applications include Foundry Nuke, Adobe After Effects, Blackmagic Design Fusion, and Autodesk Flame.

In feature film visual effects, live-action photographic plates are typically represented on disk either as raw captures of the camera data, as a log encoding of the linear camera data, often as DPX files, or as linear EXR files. Log files are typically encoded using the encoding primaries of the source camera, such as ALEXA Wide Gamut, whereas EXR files may use camera primaries or generalized primaries such as the AP0 primaries of ACES2065-1.  When working with log frames, the compositing package typically converts the images to scene-referred linear on input using a manufacturer specified transfer function, and a matrix transform to the working-space gamut if required, and converts back to the original color space on output. In its simplest form, the end to end compositing process represents a no-op, aiming to leave pixels not modified by the effect exactly as they were in the original plate. Such behavior is absolutely critical, as typically not all the shots will go through the VFX workflow, and VFX processed shots must seamlessly intercut with the remainder of the motion picture.

\subsection{Scene-Referred Linear is Preferred for Compositing}
\label{subsec:scene-referred-linear-is-preferred-for-compositing}

The benefits of scene-referred linear compositing are numerous; similar to the benefits found in rendering, shading, and lighting. All operations which blend energy with spatially neighboring pixels: motion blur, defocus, image distortion, and resizing, for example, have more physically plausible, aka realistic, results by default. Anti-aliasing works better. Light mixing preserves the appearance of the original renders. Most importantly, even simple compositing operations such as over produce more realistic results, particularly on semi-transparent elements like hair, volumetrics and fx elements.


San Francisco City Hall at Night - An HDR scene-referred linear image (top left), Defocus applied to LDR output-referred values (top right), The original scene defocused in camera (bottom left), Defocus applied to the HDR scene-referred linear data before the output transform (bottom right). All four images use the ACES sRGB Output Transform.
These image helps illustrate the energy implications of using different linearization philosophies. Pay particular attention to the appearance of the lights in the bell tower on the top of the dome. How much energy is represented in those pixels? The least realistic defocus effect is achieved by applying the filter kernel directly to the output-referred pixel values without any linearization. Since in this case there is only a small difference between the pixel values of the lights and that of diffuse white, the defocus causes the lights to be lost in the surrounding pixels.


San Francisco City Hall at Night - The original scene defocused in camera (left), Defocus applied to the HDR scene-referred linear data before the output transform (middle). Defocus applied to pseudo-linearized output-referred data (right). All three images use the ACES sRGB Output Transform.
Applying the same defocus to image data which has been converted to an approximation of scene-referred linear using a simple gamma or sRGB curve is an improvement, but still tends to de-emphasize the specular highlights. This is because even though the operation is being applied in ‘linear’, it is downstream of the output transform, and the highlights have already been tone mapped for a lower dynamic range display, restricting them to the 0-1 range, and thus do not have physically-plausible amounts of energy. Applications like Adobe After Effects work in this way by default when the “use linear” option is chosen for filters, simply removing an assumed 2.2 gamma before the filter and reapplying it afterward. Applying a defocus in scene-referred linear reveals a bokeh effect on each of the bell tower lights, mimicking the visual look of having performed this defocus during camera acquisition. This is because the pixel values are proportional to light in the original scene, and thus very bright pixels have sufficient energy to remain visible when blended with their neighbors. Other energy effects such as motion-blur achieve similar improvements in realism from working in a scene-referred linear space.

\subsection{Scene-Referred Linear Compositing Challenges}
\label{subsec:scene-referred-linear-compositing-challenges}

There are some challenges with working with high-dynamic ranges in compositing. First, filtering operators that use sharp, negative lobed kernels like as lanczos, keys, and sinc are very susceptible to ringing, i.e. negative values around highlights. While interpolatory filters such as box, gaussian, bilinear and bicubic cannot cause this artifact, they tend to introduce softening. When sharp filtering is required like lens distortions, resizing, and other spatial warping techniques a variety of approaches are useful to mitigate such HDR artifacts. A non-exhaustive set of methods are listed below

The simplest approach to mitigating overshoot/undershoot artifacts is to roll off the highlights, process the image, and then unroll the highlights back. While this does not preserve highlight energy as it has the net effect of reducing specular intensity, the results are visually pleasing. The approach is also suitable for processing images with alpha.
For images without alpha, converting to log, filtering, and converting back will greatly reduce the visual impact of overshoot and undershoot. Though log-space processing results in a gross distortion of energy, for conversions which don’t integrate large portions of the image, such as lens distortion effects for plates and not blurs, this works well.
For images without alpha, a similar method is to apply a simple invertible tonemap, filtering, and then inverting back. A method as simple as 1/(1+color) produces pleasing results by effectively weighting each sample as a function of its brightness.
Another approach to HDR filtering is to apply a simple camera flare model, where very bright highlights share their energy with neighboring pixels.

For all approaches that undertake a form of perceptual encoding, care must be taken to ensure enough precision is maintained during filtering. 16-bit floating point may not be suitable.


An image with negative black halos around highlights, produced by a Lanczos 6 resizing filter with negative lobes (left) and the image converted to a log space before resizing, thus avoiding negative black haloes (right).
Applying negative-lobed filter kernels to high-dynamic range images, may cause visually-significant ringing artifacts introducing black silhouettes around sharp specular highlights. Ringing artifacts may be avoided when using sharp filters on HDR imagery by processing in alternate color spaces, such as log, using energy-roll off approaches, or by pre-flaring highlight regions. This image demonstrates the results of the roll-off technique.

Another challenge in working with scene-referred linear data in compositing is that tricks often relied upon in integer compositing may not be effective when applied to floating-point imagery. For example, the screen operator, which remains effective when applied to matte passes, is sometimes used on RGB data to emulate a “partial add”. As the screen operator is ill-defined above 1.0, unexpected results will occur when applied to HDR data. Some compositing packages swap out screen for max when either input is outside of 0.0-1.0, but this is primarily to prevent artifacts and is not artistically helpful.  Alternative “partial add” maths such as hypotenuse are useful but do not exactly replicate the original intent of screen. Another related issue to beware of when compositing is that when combining HDR image, alphas must not go outside the range of 0 to 1. While it is entirely reasonable for the RGB channels to have any values, even negative in certain situations, compositing operators such as over, produce totally invalid results on alphas outside of the range 0 to 1.

One subtlety of working with floating point imagery is that artists must become familiar with some of the corner cases in floating-point representations: NaNs and Infs. For example, when dividing by very small values, such as during un-premultiplication, it is possible to drive a color value up high enough to generate infinity. NaNs are also frequent and may be introduced during lighting and shading inside the renderer, or during divide by zero operations. Both NaNs and Inf can cause issues in compositing if not detected and cleaned, as most image processing algorithms are not robust to their presence and generally fail in unexpected ways.

\subsection{Negative Values}
\label{subsec:negative-values}



A frame from ‘The Avengers: Infinity War”. The frame in the original Alexa LogC encoding (top), The frame transformed to ACEScc with negative values in shown as purple fringing visible in particular on the top right light (middle) and the final frame (bottom)
© 2018 MARVEL


CIE 1931 2° Standard Observer chromaticity plot of the image above
The above image contains pixel values which are positive in ALEXA Wide Gamut but negative in ACEScg as they are outside of the ACEScg gamut. This results in overly strong purple fringing and clamped values when viewed through an ACES Output Transform. The frame above the light shows the issue clearly. Even pixel values which are on or near the inside of the gamut boundary can result in artifacts. There are a number of potential approaches to solving this problem. Working in the original camera encoding gamut, ALEXA Wide Gamut in this case, is one option. Creating a technical grade for the shot which removes the artifact, and providing that as a LUT for use in VFX is another. With that option, it will be obvious if anything in compositing makes the artifact reappear.



Negative values may occur in a scene-referred linear color if the working gamut is smaller than, or even just different from, the encoding gamut of the plate. In some extreme examples, cameras have been found to produce values outside of their specified encoding gamut. That could lead to negative scene-referred linear values even without transforming away from the camera’s encoding gamut. Like NaNs and Inf values, this comes with the territory. Don’t just clip these values! These negative values can pass through some operations without causing issues and should normally be preserved when delivering VFX renders to DI. However, the results of some mathematical operations used in compositing are undefined for negative numbers and may create NaNs or Infs, or push a small negative value to a much larger one. Sometimes, where negative values will be clamped by the final display transform, it may be safe to clamp them in the comp. This is a risky strategy though. Changes to the grade may make the clamping visible, causing artifacts such as black holes in the image. It is generally preferable to use an approach such as offsetting the values before an operation to make everything positive and then subtracting the offset afterward. Remember that compositing should be a no-op for unaffected pixels, so even if a pixel value represents an unreal color in the plate, the VFX shot should be returned to DI with values representing that same unreal color, matching the original plate.

This is also something to watch out for in games. As mentioned in Section 3.4.3 Games, games may render in less precise “small” floating-point formats when an alpha channel is not needed and performance and memory bandwidth are more important than precision. It is important to remember that these floating point formats do not have a sign bit so do not support negative numbers, and thus cannot store out-of-gamut colors. If asked to do so, clipping will silently occur. Care must be taken to either choose the appropriate floating-point format, handle negative values explicitly or bring out-of-gamut values into the working gamut to avoid the silent artefacts that would otherwise occur.

\subsection{Working with Log Imagery}
\label{subsec:working-with-log-imagery}




Log plates captured for “Avengers: Infinity War” (left, right) and “The Amazing Spider-Man” (middle).
Images from “Avengers: Infinity War” are © 2018 MARVEL.  Images from “The Amazing Spider-Man” Courtesy of Columbia Pictures. © 2012 Columbia Pictures Industries, Inc. All rights reserved.
Despite scene-referred linear having benefits in many compositing operators, log-like spaces are sometimes still useful in modern compositing workflows. This is in part due to it being very straightforward to express some operations in a log space, like exposure or contrast adjustment, and in part, because some packages were designed around processing log imagery so the controls and feel of the operations are more intuitive, or familiar to artists, with log images. Modern compositing applications can apply almost every operation directly to scene-referred linear data, but it may still be useful to work with log imagery. In log space, both the linear and gamma contrast operators have reasonable performance. This is one of the reasons that colorists often prefer to work in log-like spaces. Other operations which can be useful to perform in the log domain are grain matching, pulling keys, and many spatial distortion operators. But care must be taken when using log color spaces in compositing, as not all log transforms preserve the full range of scene-referred linear code values. Most such conversions clip values above certain highlight color, below a certain shadow color, and some clip all negative values.

Scene-referred linear data is not suitable as input to a 3D LUT due to its code value distribution, range and the distribution of sensitivity in the human visual system as seen in Section 2.5.3. To address that problem, image data is often transformed into a log representation before a 3D LUT is applied. Some LUTs include this transform as a 1D shaper within the LUT file itself. See Appendix 4.4.3 for more detail.

ACES defines two standardized log working spaces, ACEScc and ACEScct, which are commonly used in grading.

\subsection{Plate Timing, Neutralization, and Grading}
\label{subsec:plate-timing-neutralization-and-grading}

When working with plate photography it is often useful to apply primary color corrections for each plate to neutralize the lighting across a sequence. During the shoot, there will inevitably be color drift across the shots, produced by events like the sun moving behind a cloud or the lighting moved to different locations from one day to the next. Using neutralized plates in lighting and compositing leads to nice gains in efficiency as light rigs and sequence standards track better. In most workflows the plate neutralization is done at the top of the compositing graph, the CG neutral elements are composited in, and then before the file is written out the neutralization is undone. This approach is often called “reverse out,” and its advantage is that the output delivery is consistent with the input plate, independent of the color neutralization process. An analogous set of operations are often applied with lens distortion being removed at the head of a comp and then reapplied as one of the last steps.
In some facilities, neutralized plates are written out to disk so everyone dealing with the images works with the neutralized result. This may be helpful to have if the background plate needs to be reflected or otherwise used in a CG render. The same is true with undistorted plates. It may be useful to write them to disk if a tracking or photogrammetry application, for example, can’t apply the same lens un-distortion as the compositing application.

It is important to draw a distinction between the correction necessary for plate neutralizations, and the potentially more sophisticated Look that is crafted by the colorist. Color corrections used for plate neutralization must be reversible, and as such are typically handled as either simple offsets in camera log space or as simple gains in linear. More complex operations such as contrast and saturation are usually avoided as these type of corrections break the linearity of the plate, are often implemented differently in different software packages and will likely make compositing more difficult. In DI grading, the full suite of color operators can be used: contrast, keys, primaries, and secondaries, as there is no need to invert this process.

For additional detail on the process of conforming sequences and neutralizing plates, see 3.6.1 Conforming.XXX

\subsection{Premultiplication}
\label{subsec:premultiplication}

Premultiplication implies that one starts with an RGB color, adds the concept of transparency, and then premultiplies by it. This isn’t actually true. Premultiplied RGB is the natural representation for color and transparency and is the native output of renderers. A nice mental model for premultiplied RGBA is that the RGB channels represent how much light is emitted from within the region of the pixel, and alpha represents how much a pixel blocks light behind it. As alpha represents the fraction of occlusion, to maintain physical correctness alpha values must be between 0 and 1. RGB channels have no such constraints and can range from -infinity to infinity.

Furthermore, there is absolutely no issue with RGB colors greater than alpha. This is a common misconception. Even when alpha is zero, a positive RGB color is a completely reasonable approximation of a physical process that emits light without occluding anything behind it. Indeed, this situation is common in emissive FX renders such as fire. Fire will also scatter and absorb light from the environment but it may in such a negligible amount that a zero-valued alpha suffices.

So how should unpremultiplied RGBA be interpreted? Unpremultiplied RGBA is most easily understood as, “If this pixel were fully opaque, how much light would have been emitted?” This representation is certainly useful in specific compositing contexts such as when pulling luma keys from CG elements, but as converting to unpremultiplied RGBA is a lossy approximation it’s prudent to only do the conversion when absolutely necessary.

For additional information on premultiplied versus unpremultiplied RGBA representations, see Porter and Duff (1984) and Blinn (1998).

\subsection{Technical Checks}
\label{subsec:technical-checks}

It is not normally practical for the workstations used for animation, lighting, VFX and compositing to show an entirely accurate preview of the final deliverable, especially if the principal master is HDR. Furthermore, while artists typically craft imagery assuming a particular viewing exposure, it is possible that the imagery will be stylistically color graded in unexpected ways in the DI grade. Therefore, care must be taken to preserve a wide dynamic range in the scene-referred linear working space, avoiding output-referred cheats and checking the result on an SDR display by pushing and pulling the exposure prior to the SDR viewing transform. Indeed, many facilities run a barrage of tests, often referred to as Quality Check (QC), Technical Checks or The Gauntlet to ensure that all completed CG and composites are robust enough to stand up to the manipulations of DI grading and that there are no artifacts which are masked by an SDR output transform that might be revealed by an HDR output transform.

The inclusion of deliverables for HDR displays makes this even more critical, as artists’ workstations may be unable to show an HDR image at all, and even if they can, they are unlikely to accurately represent the final HDR deliverable. If HDR monitoring is not possible at the desktop, having just a single shared HDR monitor to perform technical checks during the course of work can help prevent surprises that would otherwise only be discovered late during HDR finishing.

At a minimum, artists working in scene-referred linear, namely the lighter and compositor, should aspire to view their imagery at a range of exposures as part of their regular production workflow. This enables artists to have a true sense of the dynamic range in their imagery and to match computer-generated elements with similar precision. In compositing, it is also expedient to convert the imagery to the color space, quantization, and clamping that will be delivered to DI, to test how the imagery holds up to drastic color grades. Best practice is to simulate substantial exposure adjustments, as well as contrast and saturation boosts, to confirm that all portions of the imagery track in a consistent manner. This is particularly important to match across shots and sequences. It is easy to make individual shots that are self-consistent, yet fall apart when viewed side to side. Most critical is the inspection of shadow detail. “Stopping up” the final composites, shadows should have a consistent color and density both inter-shot and intra-shot. Flat black portions of the image should be avoided - lest they reappear in DI grading - and grain/noise must match between the live-action and the computer-generated imagery. Inspecting highlights, one should confirm that their intensity, color, sharpness, and flare appearance are similarly consistent. For visual effects shots that don’t modify the entire live-action plate, the portions of the image that should be unaffected by the visual effects elements should be compared against the original live-action plate. It is important to make sure that no unintended changes to black levels, color balance, grain, resolution, bit depth or lens distortion have crept into the process.

