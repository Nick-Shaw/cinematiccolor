\section{Color Matching Experiments}%
\label{sec:color-matching-experiments}

Wright & Guild 1931 2° RGB CMFs $\bar{r}(\lambda)$,$\bar{g}(\lambda)$,$\bar{b}(\lambda)$ colour matching functions are derived from the two independent experiments of Wright (1928) and Guild (1931). [3]

Both determined the chromaticity coordinates of a set of monochromatic stimuli in the wavelength range 400-700 nm using a circular 2° angular subtense bipartite field whose image is centered at the fovea of retina to avoid any participation of rod vision in the measurements.

The bottom half of the field was illuminated by the test colour stimulus to be matched and the top half was illuminated by the fixed primary stimuli at respective fixed wavelengths of 700 nm, 546.1 nm and 435.8 nm but adjustable intensity.

The observer would adjust the intensity of each of the three fixed primary stimuli until both halves of the field were colour matched.

Not all the test colour stimuli could be matched using this technique. For example, matching a yellow test colour stimulus required large amounts of red and green fixed primary stimuli. If the resulting fixed primary stimuli mixture was less yellow than the test colour stimulus, a variable amount of blue fixed primary stimulus could be added to the the test colour stimulus. For these cases, the amount of fixed primary stimulus added to the test colour stimulus was considered as a negative value.

For example, the test colour stimulus $\textbf{E}_\lambda$ at $\lambda=475$ and fixed primary stimuli values read off at $\bar{r}(475)=-0.045$, $\bar{g}(475)=0.032$ and $\bar{b}(475)=0.186$ can be expressed with the following equation: [4]

$$
\begin{equation}
\textbf{E}_{475}=-0.045\textbf{R}+0.032\textbf{G}+0.186\textbf{B}
\end{equation}
$$
The chromaticity coordinates $r(\lambda)$,$g(\lambda)$,$b(\lambda)$ of the monochromatic stimuli can then be calculated:

$$
\begin{equation}
r(\lambda)=\cfrac{\bar{r}(\lambda)}{\bar{r}(\lambda)+\bar{g}(\lambda)+\bar{b}(\lambda)}\\
g(\lambda)=\cfrac{\bar{g}(\lambda)}{\bar{r}(\lambda)+\bar{g}(\lambda)+\bar{b}(\lambda)}\\
b(\lambda)=\cfrac{\bar{b}(\lambda)}{\bar{r}(\lambda)+\bar{g}(\lambda)+\bar{b}(\lambda)}
\end{equation}
$$
with $$
\begin{equation}
r(\lambda)+g(\lambda)+b(\lambda)=1
\end{equation}
$$