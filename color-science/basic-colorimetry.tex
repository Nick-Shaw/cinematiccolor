\section{Basic Colorimetry}

\subsection{Origins of Colorimetry}

In the late 19th century, Young and Helmholtz proposed that color vision is conveyed through a limited number of receptors sensitive to different portions of the visible spectrum. Maxwell (1860) used linear algebra to prove Young–Helmholtz trichromatic theory and invented color matching experiments and modern colorimetry. The trichromatic theory explains the metamerism phenomenon: two samples matching under some lighting conditions might look different under others.


Hering proposed the opponent color theory around 1920 to explain phenomena not accounted for by the trichromatic theory. He noted that specific pairs of colors did not occur at the same time and place: a reddish-green or yellowish-blue. We, however, do see reddish-yellows and greenish-blues. He thought there were three receptor types with bipolar response to red-green, yellow-blue, and light-dark.

The Modern Opponent Colors Theory with a trichromatic first stage mediated by the cones and opponent colors encoding in the second stage. Stockman, A., & Brainard, D. H. (2010). Color vision mechanisms. The Optical Society of America Handbook of Optics, 11.1–11.104. ISBN:0071498915 - colour-science.org

The modern color vision theory,  foundational basis of colorimetry, combines both Young–Helmholtz's Trichromatic Theory and Hering’s Opponent Colors Theory: Color vision in the first stage is trichromatic and executed by the cone cells. In the second stage, upper retinal layers encode the cone cells signals into an achromatic signal V(λ) = L + M and two cone opponent chromatic signals such as RG = L - M, YB = L + M - S.
2.3.2 Measuring Light
The CIE defines colorimetry as the measurement of color stimuli based on a set of conventions. Electromagnetic radiation measurements are performed with instruments such as spectroradiometers, spectrophotometers, or tristimulus colorimeters for the case of visible radiation.

Spectroradiometry and spectrophotometry need to be described as their naming with regard to radiometry and photometry can be confusing.
Spectroradiometry measures the amount of light radiated as a function of wavelength, i.e. radiance, or irradiated, i.e. irradiance.
Spectrophotometry measures the amount of light reflected, transmitted, or absorbed as a function of wavelength, i.e. reflectance, or transmitted, i.e. transmittance.
Both disciplines measure a radiometric quantity (radiant flux) and not a photometric one contrary to what the spectrophotometry name might imply. Spectroradiometers, measuring radiance or irradiance, and spectrophotometers, measuring reflectance or transmittance, produce spectral distributions with a narrowband interval, e.g. 1nm, 5nm or 10nm.

On the other hand, tristimulus colorimeters use red, green and blue filters that emulate the HVS spectral response to light. They produce a single triplet of values by measuring wideband radiant energy and are used in digital imaging to calibrate displays. Because they mimic the HVS, tristimulus colorimeters are also subject to the metamerism phenomenon.

\subsection{Blackbody Radiation, Light sources and Illuminants}

All ordinary matter emits electromagnetic radiation when its temperature is above absolute zero. Even black holes are predicted to emit Hawking radiation, a form of blackbody radiation caused by the creation of sub-atomic particles near the event horizon. 

A blackbody or Planckian radiator is an ideal thermal radiator that absorbs all incident radiation, disregarding the wavelength, the direction of incidence or the polarization completely. In thermal equilibrium, it emits electromagnetic radiation (blackbody radiation) with a characteristic spectral distribution that depends only on its temperature and is given by Planck's Law.



When their temperature increases past 750 Kelvin, blackbodies start to emit visible electromagnetic radiation, i.e. light.

Various blackbody spectral power distributions in temperature domain 1250-10000K, note how their radiation peak slides into the visible spectrum range as their temperature increase.

Blackbodies are fundamental to both astronomy and color science allowing one to determine the color of a thermal radiator knowing only its temperature. Conversely, its temperature can be computed from its color.
In 1931, using Planck's Law, the CIE was able to unambiguously and mathematically describe a standard tungsten light source: the CIE Standard Illuminant A with a color temperature of 2856K. 


ASTM G-173 ETR Extraterrestrial Radiation (solar spectrum at top of atmosphere) at mean Earth-Sun distance compared to a Blackbody at 5778K. The differences between the curves are mostly explained by the fact that the Sun is not a perfect blackbody.
The Sun, principal light source on Earth with an effective surface temperature of 5,778K, is close to an ideal thermal radiator. Its spectrum exhibits numerous absorption and emission spectral lines, i.e. the Fraunhofer lines, resulting from the interactions of atoms, atom nuclei, and molecules with photons in its atmosphere. The Sun is not a solid body and has multiple layers with different temperatures: corona, chromosphere, photosphere, convective zone, radiative zone, and core. The photosphere, the layer of gas constituting the visible surface of the Sun, is the principal source of the radiation we receive and the region where most of the aforementioned spectral lines are forming. When solar radiation traverses the Earth atmosphere, it interacts with it, producing absorption, emission and scattering events that further modify its spectrum.

Profile image of the daylight spectrum captured with a DIY spectroscope and a CANON 5D Mark II DSLR camera. Major Fraunhofer Lines are annotated for reference. The black curve represents the luminance of the spectrum at each wavelength. The white curve is a reference spectrum concurrently captured with an Ocean Optics STS-VIS spectrometer. While the spectroscope has a very high resolution, the overall curve shape is dramatically different to that of the spectrometer one: the DSLR spectral sensitivities, its infrared filter, and reflections in the spectroscope tube are all contributing to alter the curve, making it unsuitable for precise measurements.
CIE Illuminants B and C were intended to respectively model noon sunlight (correlated color temperature (CCT) of 4874K) and average daylight (CCT of 6774K). They are realised by combining a light source modeling CIE Standard Illuminant A with Davis-Gibson liquid filters from David and Gibson (1931). Because of the low illuminance levels achieved with such light sources and their poor performance in the near-ultraviolet region (which is essential for fluorescent materials), they were deprecated in favor of the CIE Illuminant D Series created by Judd, MacAdam, and Wyszecki (1964). 

The CIE Illuminant D Series models a broad range of daylight phases inclusive of direct sunlight, sunlight with cloud coverage, full overcast coverage, and blue skylight. For standardization purposes, "CIE Standard Illuminant D Series D65 should be used in all colorimetric calculations requiring representative daylight unless there are specific reasons for using a different illuminant" (ISO 11664-2:2007(E)/CIES014-2/E:2006).

CIE Illuminants B, C, and CIE Standard Illuminant D Series D65. Note how the CIE Illuminants B and C curves are different to the CIE Standard Illuminant D Series D65 curve in the 300nm-380nm near-ultraviolet region.
From a terminology standpoint, it is meaningful to understand that illuminants and light sources are different:
An illuminant is a standardized table of values or mathematical function representing an ideal light source.
A light source is a physical emitter of visible radiant energy.

\subsection{Reflectance, Transmittance and Absorptance}

Emitted light interacts with surfaces of the world and is either reflected, transmitted or absorbed. The measured radiant flux is the summation of the light reflection, transmission, and absorption at each wavelength such as:
Φ(λ) = R(λ) + T(λ) + A(λ)
where Φ(λ) is the measured radiant flux, R(λ), T(λ) and A(λ) are respectively the surface reflectance, transmittance, and absorptance. They are ordinarily expressed as fractions of Φ(λ) in range 0-1.

Clover leaf reflectance, transmittance, and absorptance spectra from LOPEX93, note that the wavelengths represented are purposely reaching near-infrared: even though invisible, the spectra have values in that portion and exhibit a lot of variation.
Because they are also a function of the illumination and viewing geometry, i.e. the angle between the light incident at the surface and the angle of the instrument, a complete description requires capturing the Bidirectional Reflectance Distribution Function (BRDF) or Bidirectional Transmittance Distribution Function (BTDF). High-quality BxDF data capture is not only difficult, but its usage is also non-trivial thus for practical purposes the CIE defines 4 standard illuminations and viewing geometries. They are described in CIE 015:2004 Colorimetry, 3rd Edition. 

\subsection{Standard Observers}

The perception of incident light to the HVS varies from one observer to another. Thus to avoid ambiguity when characterizing color, the CIE standardized various functions for use with colorimetric calculations.

The first one, the CIE 1924 Photopic Standard Observer luminous efficiency function V(λ), a fundamental function modeling the wavelength dependent sensitivity of the HVS, was established in 1924.

V(λ) - CIE 1924 Photopic Standard Observer modeling the wavelength dependent sensitivity of the HVS.

It was derived for a 2° angular subtense viewing field from several independent experiments whose results were weight assembled by Gibson and Tyndall (1923).

The first row shows the computed swatches and spectral distributions of two synthetic green and red LED lights with equal radiant power. In the second row, the red LED light luminous flux has been scaled to match that of the green LED light.

To an observer viewing a red light and a green light with equal radiant power, the green light appears brighter than the red: in the above figure their luminous fluxes are 40500lm, and 3500lm respectively. If the red light luminous flux is scaled to match that of the green light, then both lights are perceived to have similar brightness. The consequence of this scaling is that the red light radiant power increased by one order of magnitude.

In the late 1920's, Wright (1928) and Guild (1931) independently conducted a series of color matching experiments to quantify the color ability of an average human observer which laid the foundations for the specification of the CIE XYZ color space. The results obtained were summarized by the Wright & Guild 1931 2° RGB― color Matching Functions (CMFs): they represent the amounts of three monochromatic primary colors R, G, B needed to match the test color at a single wavelength of light.


Wright & Guild 1931 2° RGB― CMFs present a negative lobe.

Wright & Guild 1931 2° RGB― CMFs present negative values that are inconvenient for various reasons: they make RGB tristimulus values computation more difficult, requiring separately summing products with positive and negative signs and then a final differencing of the sums. Computation of photometric quantities like luminance is more laborious for similar reasons, and finally, the development of direct-reading tristimulus colorimeters is challenging because of the sign change.

These reasons lead the CIE to transform the Wright & Guild 1931 2° RGB CMFs into a new set of functions based on new primary stimuli X, Y, Z: The CIE 1931 2° Standard Observer XYZ― color matching functions. When conceiving the linear transformation that converts Wright & Guild 1931 2° RGB― CMFs into the CIE 1931 2° Standard Observer XYZ CMFs, the CIE ensured that the Y― function was equal to the CIE 1924 Photopic Standard Observer luminous efficiency function V(λ) and that all the functions were positive.
 

The CIE 1931 2° Standard Observer, a linear transformation of Wright & Guild 1931 2° RGB CMFs designed to remove the negative lobe of the later.
 
Among other factors such as age, color vision sensitivity is affected by the angle subtended by the objects being observed. In the 1960s it was found that cones were present in a broader region of the eye than that initially covered by the experiments that lead to the CIE 1931 2° Standard Observer specification. As a result, color computations done with the CIE 1931 2° Standard Observer do not always correlate to visual observation.

In 1964, the CIE defined an additional standard observer: the CIE 1964 10° Standard Observer derived from the results of Stiles and Burch (1959) and Speranskaya (1959) investigations with large-field color matching experiments.


The CIE 1964 10° Standard Observer corrects some deficiencies of the CIE 1931 2° Standard Observer, especially in the blue wavelengths. It should be used when dealing with a field of view of more than 4°.

The CIE 1964 10° Standard Observer is believed to be a better representation of the spectral response of human vision and recommended by the CIE when dealing with a field of view of more than 4°. For example, the CIE 1964 10° Standard Observer is commonly used in color formulation and color quality control whereas the CIE 1931 2° Standard Observer tend to be used with applications that deals with small samples or samples viewed at great distance.

Virtually all computer graphics applications use the CIE 1931 2° Standard Observer. Motion picture color management also adopts it as it is suited for computer graphics and imagery:
Interface elements, e.g. icons, color pickers and swatches, and image inner features tend to be small and subtend a small angle. Hunt (2004) says that "in colour reproductions, the interest generally lies much more in patches of colour of about 2° angular size than 10°, and the 1931 CIE data may therefore be used with confidence."
The CIE 1931 2° Standard Observer only involves the fovea where the cones are present, i.e. color vision, and great care was taken to avoid rod intrusion when the CIE 1964 10° Standard Observer data was assembled with the consequence that similar care must be taken when evaluating 10° subtending fields.
Recently, the CIE TC 1-36 technical committee report (CIE 170-1:2006, 2006) established the CIE 2012 10° Standard Observer as the new physiologically relevant fundamental CIE color matching functions. They are a linear transformations of Stockman & Sharpe 10° Cone Fundamentals LMS10 spectral sensitivity functions which can be used to derive any CMFs given physiological parameters such as the age of the observer and the field size.

The transition to those new CMFs might happen in the future but there are open questions on how to migrate dataset that are not spectrally defined, e.g. what are the chromaticity coordinates of the sRGB color space in the CIE 2012 10° Standard Observer?

\subsection{CIE XYZ Tristimulus Values and Metamerism}

\subsubsection{CIE XYZ Tristimulus Values}

Spectral radiant energy is converted into CIE XYZ tristimulus values by integrating the product of the spectral distribution of a sample with the spectral power distribution of a light source (or illuminant) and with the CMFS. Their values for the color of a surface with spectral reflectance β(λ) under an illuminant of relative spectral power S(λ) are calculated using the following integral equations:


Equation x.x of the computation of X tristimulus value is illustrated in figure x.x where the reflectance of a sample of sand is multiplied by CIE Standard Illuminant D Series D65, the color matching function X― and the normalization factor k.

Integration of the spectral reflectance distribution of a sample of sand for X tristimulus value. Note that this process must be repeated for Y― and Z― shown in dashed green and blue lines to obtain the corresponding Y and Z tristimulus values.

\subsubsection{Metamerism}

The conversion of spectral radiant energy into CIE XYZ tristimulus values reduces complex light color information spanning many wavelengths into three sensory quantities. From a mathematical standpoint, the tristimulus conversion equations X.X.X are demonstrating the phenomenon: the integrals represent the area under the multiplied spectral distribution curves of the sample, light, and CMFS. However, an infinite number of light and sample curve combinations yield the same area: tristimulus values are multi-valued, and metamerism occurs because of the lack of injectivity. A consequence of this reduction is that two samples viewed under the same illumination conditions can produce an identical cone cells response and thus yield matching CIE XYZ tristimulus values. The two samples are said to be metamers, and changing the illumination conditions might result into a metameric failure: this is the metamerism phenomenon.


Metameric failure example, the two spheres use generated metameric spectral reflectance distributions that yield matching color under CIE Illuminant E but not under CIE Standard Illuminant A.

\subsection{CIE xyY Color Space and Chromaticity Diagram}

It is convenient to use a 2D representation of colors where luminance is separated from chroma. The CIE xyY color space is constructed through a projective transformation that isolates the Y luminance axis of the CIE XYZ color space and yields a tuple of chromaticity coordinates (x, y):

The CIE 1931 Chromaticity Diagram is a 2D projection of the volume of all the colors seen by the CIE 1931 2° Standard Observer along the Y luminance axis of the CIE xyY color space.

The CIE 1931 Chromaticity Diagram and Pointer (1980)'s Gamut, a gamut of real surface colors. Note that similarly to the figures showcasing the visible spectrum, the colors are accurate within the limits of what the display can represent.

The curved edge encompassing the colors is known as the Spectral Locus and is composed of physically realizable monochromatic colors, each label on the diagram is such a color. The encompassed colors are mixtures of the edge monochromatic colors. The region outside does not correspond to physically-possible colors, those chromaticity coordinates outside the spectral locus are referred to as "imaginary". They are often useful for mathematical encoding purposes but are not realizable in any display system. The line closing the horseshoe shape is known as the line of purples.

The volume (or area) of colors either "present in a specific scene, artwork, photograph, photomechanical, or other reproduction; capable of being created using a particular output device and/or medium" is defined by the CIE as the color gamut. Color gamuts are generally represented on the chromaticity diagram as flat, two-dimensional surfaces with uniform luminance. They are specified with a set of tuple of chromaticity coordinates (x, y). The chosen luminance value, often 1, is arbitrary and depends on the use case. For analysis and visualization purposes, it is valuable to draw multiple variants of a color gamut at different luminance values as its surface boundary changes accordingly. By extension, color gamuts such as Pointer (1980)'s gamut of real surface colors, specified with a set of triplets of CIE xyY color space coordinates are representing three-dimensional volumes.

\subsection{Perceptually Uniform Color Spaces and Color Difference}

Although it is not immediately apparent, the chromaticity diagram is a very poor representation of our color perception. Distances between colors in the CIE xyY color space do not directly relate to their apparent perceptual differences. Two adjacent colors may be perceived as different, while colors far apart may be indistinguishable. The degree to which a given color space accurately maps geometric distance to perceptual distance is referred to as its level of perceptual uniformity.


MacAdam (1942) ellipses, i.e. indistinguishable color regions, plotted in the CIE 1931 chromaticity diagram and the CIE 1976 UCS chromaticity diagram with improved perceptual uniformity: the ellipses are rounder. Note how the green region of the CIE 1931 chromaticity diagram is highly stretched compared to the same region in the CIE 1976 UCS chromaticity diagram or compared to the blue region in both diagrams. An ideal color space would transform the ellipses into circles. Note that the ellipses axes are 10 times their actual length and that it is possible to generate in-between ellipses by interpolation. Data from Table 2(5.4.1) in Wyszecki, G., & Stiles, W. S. (2000). Color Science: Concepts and Methods, Quantitative Data and Formulae. Wiley. ISBN:978-0471399186

Perceptual uniformity can be a desirable quality for a color space: it supports accurate color differences and lightness prediction. A perceptually uniform color space improves performance of various image processing algorithms such as lossy image compression, denoising, segmentation or device characterisation, image quality modeling and color appearance modeling.

Gamut mapping, the process by which colors lying outside a gamut are mapped inside it, is preferably performed in a perceptually uniform color space because the correlation between the perceptual attributes is reduced and thus hue is minimally affected when the lightness and chroma attributes are individually adjusted to reduce the gamut size.

The non-uniformity of the CIE 1931 Chromaticity Diagram led to the adoption by the CIE in 1960 of the uniform color space devised by MacAdam (1937). The CIE 1960 Uniform Color Space (UCS) is nowadays primarily used to represent the correlated color temperature (CCT) of light sources or illuminants as the isolines are perpendicular to the Planckian locus. The isolines represent delta uv: for a given CCT and its corresponding color, increasing or decreasing delta uv will bias the color toward green or magenta. The tint control in white balancing algorithms is effectively delta uv scaled by a constant.
 
CIE 1960 UCS Chromaticity Diagram with Planckian Locus and perpendicular Iso-Temperature lines.

Conversion from tristimulus values to CIE 1960 UCS color space UVW values is performed as follows:
UVW = ⅔ × X, Y, ½ × (-X + 3 × Y + Z)
The chromaticity coordinates uv are then obtained by a perspective projection similarly to the one of CIE xyY color space:
uv =U / (U + V + W), V / (U + V + W)

The CIE 1976 UCS Chromaticity Diagram, based on CIE L*u*v* color space, superseded the CIE 1960 UCS Chromaticity Diagram.

CIE 1976 UCS Chromaticity Diagram based on CIE L*u*v* color space.

The conversion from tristimulus values to CIE L*u*v* color space is longer than CIE 1960 UCS and is found in the Appendix.
Other uniform color spaces beyond CIE L*u*v* have been defined: the IPT color space by Ebner and Fairchild (1998) is excellent at predicting perceived hue. CAM16-UCS by Li, Li, Wang, Zu, Luo, Cui, Melgosa, Brill and Pointer (2017) is a uniform color space based on CAM16 that offers good general performance. The ICtCp color space by Lu, Pu, Yin et al. (2016) and the JzAzBz color space by Safdar, Cui, Kim, and Luo (2017) extend perceptual uniformity to high dynamic range and adopt the PQ curve as the basis for Lightness prediction.

The need for quantifying how much two color samples are different was one of the incentives for the CIE to design a perceptually uniform color space and ultimately adopt both the CIE L*a*b* and CIE L*u*v* color spaces in 1976. Color differences are measured as the Euclidean distance between a reference color sample and a test color sample in the CIE L*a*b* color space as follows:
∆E*76 = √ ((L*r - L*t)2 + (a*r - a*t)2  + (b*r - b*t)2)
where ∆E*76 is the color difference, L*r, a*r, b*r , and L*t, a*t, b*t are the reference and test color sample. 

While the CIE L*a*b* color space features decent perceptual uniformity, it was deemed unsatisfactory when comparing some pair of colors, compelling the CIE into improving the metric with the CIE 1994 (∆E*94) and subsequent CIE 2000  (∆E*00)  quasimetrics. It is appropriate to measure color differences in the CAM16-UCS, ICtCp and JzAzBz color spaces using Euclidean distance.

\subsection{Additive RGB color spaces}

Additive RGB color spaces are the color representation that most people interact with on a daily basis; our display technology is based on the additive mixing of 3 red, green and blue primaries.

The first RGB color space, CIE RGB, was created by Wright & Guild (1931) when they performed their color matching experiments. The three monochromatic primary colors R, G, B used to match a test color at a single wavelength of light once plotted on the CIE 1931 Chromaticity Diagram, are located on the spectral locus and respectively have the following wavelengths: 700.0 nm, 546.1 nm, and 435.8 nm. sRGB is the ubiquitous RGB color space created by Hewlett Packard and Microsoft in 1996, standardized as IEC 61966-2-1:1999, in an attempt to define a standard color space for use with monitors, printers, and on the Internet. Many other RGB color spaces exist and are reviewed in the Appendix. This section of the document focuses instead on the components of an RGB color space and provides some disambiguation of terminology.

It is common to see operators in digital content creation (DCC) applications or functions in various game engines, libraries, color processing tools labeled as "linear to sRGB". The consequence of poor wording is that people often don't understand that sRGB is probably more than a "linear to sRGB" operator or function. Some resources purport to explain how sRGB and linear color spaces differ.
First and foremost, while the CIE RGB and sRGB color space gamuts are intrinsically representing radiometrically linear light values, their color component transfer functions are radically different: CIE RGB has no defined encoding and decoding color component transfer functions; they are considered to be linear; however sRGB adopts non-linear color encoding and decoding component transfer functions following a power law.

The confusion comes from the fact that generally, people don't mention which component of the additive RGB color space they are referring to. The following three components are required to fully specify an additive RGB color space:

Primaries
White point
Color Component Transfer Functions

Noticeably, the ISO 22028-1:2016 Standard defines an additive RGB color space as follows:

	Additive RGB color space

Colorimetric color space having three color primaries (generally red, green and blue) such that CIE XYZ tristimulus values can be determined from the RGB color space values by forming a weighted combination of the CIE XYZ tristimulus values for the individual color primaries, where the weights are proportional to the radiometrically linear color space values for the corresponding color primaries.

Note 1 to entry: A simple linear 3 × 3 matrix transformation can be used to transform between CIE XYZ tristimulus values and the radiometrically linear color space values for an additive RGB color space.

Note 2 to entry: Additive RGB color spaces are defined by specifying the CIE chromaticity values for a set of additive RGB primaries and a color space white point, together with a color component transfer function.

\subsubsection{Primaries}

The primaries’ chromaticity coordinates define the gamut that can be encoded by a given RGB color space. It is essential to understand that while commonly represented as triangles on a Chromaticity Diagram, RGB color space gamuts define the boundaries of an actual 3D volume within the CIE xyY color space. The shape of the volume is not simply an extruded triangle because each primary reaches unity at a different luminance value. From that point the luminance can only rise by increasing one or both of the other channels, and consequently decreasing saturation. Maximum luminance is only achieved when the three channels reach unity, which corresponds to the achromatic point at the top of the volume.
A gamut, per definition, only contains physically realizable colors but RGB color space primaries can be located outside of the spectral locus, in such cases, the gamut is the intersection between the given RGB color space volume and the volume of physically realizable colors.

sRGB color space gamut visualized in CIE xyY color space. The shape of the volume is the result of the transformation of an RGB unit cube to CIE XYZ color space followed by a transformation to CIE xyY color space.


sRGB, DCI-P3 and BT.2020 RGB color space gamuts in the CIE 1931 chromaticity diagram.

\subsubsection{White Point}

The white point is defined by the CIE as the "achromatic reference stimulus in a chromaticity diagram that corresponds to the stimulus that produces an image area that has the perception of white". Any colors lying on the neutral axis line defined by the white point and its vertical projection toward the bottom of the RGB color space gamut, no matter their Luminance, are neutral to that RGB color space and thus achromatic.

Various RGB color spaces white points in the CIE 1960 UCS chromaticity diagram, note that the ACES white point is not precisely equal to CIE Illuminant D Series D60.

Typical white points are chosen as CIE Standard Illuminants. In 1953, the National Television System Committee (NTSC) published a standard for color television that used CIE Illuminant C. As noted in section 2.3.3, CIE Illuminant C was superseded by the CIE Standard Illuminant D Series D65 and became the illuminant for the PAL and SECAM analog encoding systems for color television in the 1960s.
In the 1990s, desktop publishing CRT color monitors often adopted a white point close to CCT 9300 K which was deemed too high by the professional printing industry and graphic arts to suitably soft proof images. Printing workflows use CIE Illuminant D Series D50 (ISO 3664:1975) as it is somewhere between average daylight and incandescent light. CIE Illuminant D Series D50 is the illuminant adopted by the International Color Consortium (ICC) established in 1993 by eight imaging companies "for the purpose of creating, promoting and encouraging the standardization and evolution of an open, vendor-neutral, cross-platform color management architecture and components."

Two white points were suggested as an alternative to the high CRT CCT: CIE Illuminant D Series D50 and CIE Standard Illuminant D Series D65. Images displayed on monitors calibrated to CIE Illuminant D Series D50 were deemed dull and yellow in dim to average viewing conditions.

Following on from PAL and SECAM color television, and also as a consequence of the broad adoption of the sRGB IEC 61966-2-1:1999 standard, most modern computer display systems use CIE Standard Illuminant D Series D65.

Digital cinema as per the recommendation of the Digital Cinema Initiative (DCI) adopts the DCI-P3 white point based on a Xenon arc lamp spectral power distribution. ACES uses a white point close to CIE Standard Illuminant D Series D60: its lower CCT makes it more pleasing for dark viewing conditions typical of theatrical exhibition.

\subsubsection{Color Component Transfer Functions}
The color component transfer functions (CCTFs), defined as "single variable, monotonic mathematical function applied individually to one or more colour channels of a colour space" by  ISO 22028-1:2016 Standard generally perform the mapping between linear light components and a non-linear R'G'B' component signal value, most of the time for coding optimization and bandwidth performance and/or to improve the visual uniformity of a color space.

\paragraph{Encoding Color Component Transfer Functions}

The Encoding CCTF, commonly an Opto-Electronic Transfer Function (OETF) or inverse Electro-Optical Transfer Function (EOTF), maps estimated linear light components in a scene to a non-linear R'G'B' component signal value.

sRGB, Rec. 709 and Cineon Encoding color Component Transfer Functions, note how the Cineon curve imposes a substantial compression to the input domain allowing it to encode a wider dynamic range.

Typical OETFs are expressed by a power function with an exponent between 0.4 and 0.5. They can also be defined as piece-wise functions such as the sRGB or Rec. 709 OETFs. OETFs are necessary for the faithful representation of images and perceptual coding in relation to display non-linear response and the non-linearity of the human visual system.

Digital cinema cameras usually adopt logarithmic encoding functions such as Sony S-Log, ALEXA LogC or Canon C-Log allowing storage of a much wider dynamic range from the original scene compared to a generic Rec. 709 HDTV camera. To some degree, they are considered as flavors of the Cineon encoding found in the Kodak Cineon System, an early 1990s computer-based digital film system. The main difference between digital cinema cameras encodings and Cineon is that the latter is a logarithmic encoding directly corresponding to the optical density of the film negative. In order to know the relationship between a Cineon encoding and scene light, it is necessary also to know the characteristic curve of the film stock.


\paragraph{Decoding Color Component Transfer Functions}

The Decoding CCTF, commonly an EOTF or inverse OETF, maps a non-linear R'G'B' video component signal to a linear light components at the display. It describes how a display, such as a TV or a projector, responds to an incoming electrical signal and how it converts code values into visible light.


sRGB, Rec. 1886 and Gamma 2.2 Decoding color Component Transfer Functions. Note that the sRGB transfer function used as an EOTF is subject to debate as seen in section 4.1.3.

Typical EOTFs are expressed by a power function with an exponent between 2.2 and 2.6 such as BT.1886 for Standard Dynamic Range (SDR) television or a piece-wise function for display using the exact sRGB transfer function as EOTF. High Dynamic Range (HDR) television as standardized by BT.2100 adopts Hybrid Log-Gamma (HLG) or PQ EOTFs.

A listing of commonly used OETFs and EOTFs is included in the Appendix with the specific formulae and constants used for each.

